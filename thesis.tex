\documentclass[11pt,oneside]{book}

\usepackage{amsthm, amsmath, amssymb, hyperref, mathrsfs, graphicx, tabularx, multirow}

\DeclareMathOperator*{\argmax}{arg\,max}
\DeclareMathOperator*{\argmin}{arg\,min}
\DeclareMathOperator{\permindex}{index}
\DeclareMathOperator{\Inv}{Inv}
\DeclareMathOperator{\meet}{\wedge}
\DeclareMathOperator{\bigmeet}{\bigwedge}
\DeclareMathOperator{\join}{\vee}
\DeclareMathOperator{\bigjoin}{\bigvee}

\theoremstyle{plain}
\newtheorem{theorem}{Theorem}
\newtheorem{lemma}[theorem]{Lemma}
\newtheorem{proposition}[theorem]{Proposition}
\newtheorem{corollary}[theorem]{Corollary}

\theoremstyle{definition}
\newtheorem{definition}{Definition}[section]
\newtheorem{conjecture}{Conjecture}[section]
\newtheorem{example}{Example}[section]

\theoremstyle{remark}
\newtheorem*{remark}{Remark}
\newtheorem*{note}{Note}
\newtheorem{case}{Case}
\newtheorem{claim}{Claim}

\newtheoremstyle{customtheoremstyle}
	{\topsep}
	{\topsep}
	{\itshape}
	{}
	{\bfseries}
	{.}
	{.5em}
	{\thmname{#3}}

\theoremstyle{customtheoremstyle}
\newtheorem{customtheorem}{Custom}

% \includeonly{step-three}

\begin{document}

\frontmatter

\title{A Generalized Probabilistic Gibbard-Satterthwaite Theorem}
\author{Jonathan Potter \\ jmp2909@rit.edu}
\date{\today}
\maketitle

\tableofcontents
% \listoffigures
% \listoftables

\mainmatter

%!TEX root = thesis.tex


\chapter{Introduction}

	In this thesis we endeavor to extend the results of Friedgut, Kalai, and Nisan \cite{friedgut2008elections} who proved that social choice functions can be successfully manipulated by random preference reordering with non-negligible probability. However, there are two main restrictions on their results: the social choice function must be neutral, and the election must have at most 3 alternatives. We attempt to remove the later restriction in order to generalize the results to elections with any number of candidates.

	Our proof draws upon many aspects of Friedgut, Kalai, and Nisan's proof. Their proof is done in three steps, with the first two steps being already written in general terms, while the third is restricted to 3 alternatives. Therefore we need only generalize the third step. We rely heavily on lattice theory and combinatorics to prove this generalization.


\section{Importance}

	It is obvious, and widely accepted that election systems are important to society. They are essential to democracy which is the foundation of many nations' governments, and are also used in many non-political situations---anywhere a group of independent agents needs to come to a consensus. They are used in schools, electing board members for a business, and stock holders voting on issues affecting a company. Election systems are not even wholly reserved to humans. Elections can be used by artificial intelligence systems when a group of agents needs to make a decision \cite{ephrati1991clarke, ephrati1993multi, pennock2000social, dwork2001rank, fagin2003efficient}, and they can be used in internet page ranking algorithms for search engines \cite{chevaleyre2007short}.

	However, it is less clear that manipulation in elections---especially random manipulation---is important, so we will attempt to describe its importance by briefly explaining the history behind it.

	One obvious criteria for a good election system is fairness \cite{chevaleyre2006issues}, and it is generally accepted that the winning candidate should, on the whole, represent the will the constituents'. It is easy to recognize a fair election system if there are only two candidates: the candidate who is preferred by the majority of voters should win. But with a larger number of candidates, determining the fairness of an election system is more difficult.

	Marquis de Condorcet, was one of the first people to study (academically) the issue of fairness in election systems. He proposed that the winning candidate be the candidate who would win a head-to-head election against each of the other candidates, and such a winner is known as the \emph{Condorcet winner}. Unfortunately, Condorcet also proved that a Condorcet winner does not always exist. Nevertheless, this criterion for fairness, called the \emph{Condorcet criterion}, was one of the first formal fairness criterion, and is still widely used today.

	In 1950, Kenneth Arrow, an American economist who was interested in the fairness of social welfare functions, made a large contribution to the field of social choice theory with his impossibility theorem \cite{arrow1950difficulty, arrow1963social}. This theorem demonstrates that no social welfare function can ``fairly'' convert the preferences of voters into a society-wide preference list by showing that no social welfare function can satisfy the following criteria (which will be further described in the next chapter): unrestricted domain, independence of irrelevant alternatives, unanimity, and non-dictatorship.

	One of the great enemies of fairness in election systems is \emph{manipulation} (or strategic voting or tactical voting). Manipulation is when an individual purposefully misrepresents his preferences hoping to achieve a more favorable outcome in the election. One way to avoid manipulation would be to devise a voting rule that is non-manipulable. Unfortunately, the Gibbard-Satterthwaite theorem states that every voting rule which is not a dictatorship, and under which any alternative can win, is subject to manipulation \cite{gibbard1973manipulation, satterthwaite1975strategy, duggan2000strategic}. This means that we cannot make manipulation impossible via a cleverly devised voting rule.

	In an attempt to circumvent the Gibbard-Satterthwaite theorem, Bartholdi, Tovey, and Trick studied the computational difficulty of finding a winner for various voting rules. For example, they showed that the Dodgson method mentioned above \cite{dodgson1876method} is actually infeasible to manipulate for the simple reason that figuring out the winner of the election is NP-hard. Therefore, it is not sufficient for a desirable voting rule to be hard to manipulate: it must also be also be efficient to determine a winner.

	Many others have followed in the vein of searching for a computational barrier to manipulation, but the majority of these results deal with the worst-case complexity of manipulation. In 2006, work by Conitzer and Sandholm \cite{conitzer2006nonexistence} along with that of Procaccia and Rosenschein \cite{procaccia2006junta} showed that while manipulation can be hard in the worst case, it is often much easier in the average case. In the next few years more work was done to make this concern even more well-founded \cite{procaccia2007average, erdelyi2007approximating}.

	In 2008, Friedgut, Kalai, and Nisan \cite{friedgut2008elections}, instead of studying worst-case manipulation, performed a probabilistic analysis of random manipulation. That is, instead of a voter intelligently manipulating an election, which can be difficult in terms of worst-case complexity, he simply chooses his manipulation randomly (if his most preferred candidate is not winning already). They proved that even a random manipulation will succeed with non-negligible probability. This is significant because no matter how hard it is in the worst-case to find a profitable manipulation, if it is trivial to find a random manipulation, that could be enough. These are the results we hope to extend.


\section{Difficulty}

	The difficulty of this problem can be seen by the recent work done in generalizing the results of Friedgut, Kalai, and Nisan. First, its difficulty can be seen by Friedgut, Kalai, and Nisan themselves failing to generalize it, both in the original paper \cite{friedgut2008elections}, and also later when they removed the neutrality constraint \cite{friedgut2011quantitative}. If it were an easy task, they would have done it from the outset.

	In addition, other authors have done work along the same lines, but still without coming up with a general result. In 2008 Xia and Conitzer were able to prove a similar theorem for any number of candidates, but instead of neutrality they assumed 5 other conditions for the voting rule \cite{xia2008sufficient}:

	\begin{itemize}
		\item Homogeneity
		\item Anonymity
		\item Non-imposition
		\item Canceling out
		\item Stability
	\end{itemize}

	These conditions are formally defined and explained in Chapter 4: Related Work, and also by Xia and Conitzer.

	These conditions are stricter than the neutrality assumption of Friedgut, Kalai, and Nisan, in the sense that they do not capture all of the ``common'' voting rules, e.g. Bucklin.

	Around the same time Dobzinski and Procaccia published complementary results for two voters and social choice functions satisfying unanimity (the Pareto principle) \cite{dobzinski2008frequent}. They proved the following:

	\begin{theorem}[Dobzinski and Procaccia]
		Let $f$ be a Pareto-optimal SCF and let $n = 2$, $m \ge 3$, and $\delta < \frac{1}{32m^9}$. If $f$ is $\delta$-strategyproof then $f$ is $16m^8 \delta$-dictatorial.
	\end{theorem}

	The fact that all of these authors worked on the same problem over multiple years and were unable to achieve a general result speaks to its difficulty.


\section{Independent Work}

	Unfortunately for us, but fortunately for the field of social choice theory as a whole, Isaksson, Kindler, and Mossel \emph{have}, independently during the writing of this thesis, published a brilliant generalization of the original theorem of Friedgut, Kalai, and Nisan and even improved slightly upon the results \cite{isaksson2010geometry}. Translating their results into the terminology we have been using, they proved that for a neutral social choice function $f$ with $m \ge 4$ alternatives and $n$ voters that is $\epsilon$-far from dictatorship, a uniformly chosen profile will be manipulable with probability at least $2^{-1} \epsilon^2 n^{-4} m^{-6} (m!)^{-3}$.

	Later Friedgut et al. removed the neutrality constraint from their original theorem, and added an author \cite{friedgut2011quantitative}.

	Finally, Mossel and R\'{a}cz \cite{mossel2011quantitative} took ideas from these two proofs and created a unified proof with the same results as Isaksson, Kindler, and Mossel, but without the neutrality constraint.

	Though these results have independently achieved the goals we set out with, we believe that our work is still useful. At the very least ours simply stands as an alternate proof. However, our proof has the benefit that it uses very similar techniques to those of the original proof of Friedgut, Kalai, and Nisan, and additionally we believe that our proof is much simpler and more easily understood.


\section {Proof Summary}

	The proof we are generalizing is broken up into three steps. In the original paper they are called Step 1, Step 2, and Step 3, but in \cite{friedgut2011quantitative} they are called the following respectively:
	\begin{enumerate}
		\item Applying a quantitative version of Arrow's impossibility theorem
		\item Reduction from low manipulation power to low dependence on irrelevant alternatives
		\item Reduction from low manipulation power to low dependence on irrelevant alternatives.
	\end{enumerate}
	In this thesis we will refer to them as Step 1, Step 2, and Step 3.

	In the original paper, Friedgut, Kalai, and Nisan were able to generalize Step 1 and Step 2 as follows:

	\begin{lemma}[Generalized Step 1]
		For every fixed $m$ and $\epsilon > 0$ there exists $\delta > 0$ such that if $F = f^{\otimes \binom{m}{2}}$ is a neutral IIA GSWF over $m$ alternatives with $f : \{0,1\}^n \rightarrow \{0,1\}$, and $\Delta(f, DICT) > \epsilon$, then $F$ has probability of at least $\delta \ge (C\epsilon)^{\lfloor m/3 \rfloor}$ of not having a Generalized Condorcet Winner, where $C > 0$ is an absolute constant.
	\end{lemma}

	\begin{lemma}[Generalized Step 2]
		For every fixed $m$ there exists $\delta > 0$ such that for all $\epsilon > 0$ the following holds. Let $f$ be a neutral SCF among $m$ alternatives such that $\Delta(f, DICT) > \epsilon$. Then for all $(a,b)$ we have $M^{a,b}(f) \ge \delta$.
	\end{lemma}

	Therefore, we focus on generalizing Step 3. The original Step 3 was:

	\begin{lemma}[Non-General Step 3]
		For every SCF $f$ on $3$ alternatives and every $a,b \in A$, $M^{a,b} \le \sum_i M_i \cdot 6$
	\end{lemma}

	And our generalization is:

	\begin{lemma}[Generalized Step 3]
		For every SCF $f$ on $m$ alternatives and every $a,b \in A$, $M^{a,b} \le \sum_i M_i \cdot m!$
	\end{lemma}

	When we put together all 3 generalized steps we get our main result:
	\begin{theorem}[Main Result]
		There exists a constant $C > 0$ such that for every $\epsilon > 0$ the following holds. If $f$ is a neutral SCF for $n$ voters over 3 alternatives and $\Delta(f, g) > \epsilon$ for any dictatorship $g$, then $f$ has total manipulability: $\sum^n_{i=1} M_i(f) \ge \frac{(C\epsilon)^{\lfloor m/3 \rfloor}}{m!}$.
	\end{theorem}


\section{Structure of the Remaining Chapters}

	\begin{description}
		\item[Chapter 2: Preliminaries] In the next chapter we introduce the preliminaries. These include formal definitions and notation to serve as a reference for use in the rest of the thesis. The preliminaries are often elementary but provide a technical foundation for the following work.

		\item[Chapter 3: Background] Here we give some background information on the field of social choice theory and describe how it has evolved to lead to the problem we are solving.

		\item[Chapter 4: Related Work] In the related work chapter we will describe, in a moderate amount of detail, the results and methods of various other authors relating to the work of Friedgut, Kalai, and Nisan and, hence, to ours.

		\item[Chapter 5: Results] This is the technical portion of the thesis in which we prove some foundational lemmas and eventually build up a proof of our main result.
	\end{description}

%!TEX root = thesis.tex

\chapter{Preliminaries}

	\begin{definition}
		A \emph{permutation} of a set $X$ is a bijective function from $X$ to $X$.
	\end{definition}

	\begin{definition}
		We use $L(X)$ to denote the set of all total orders over $X$.
	\end{definition}

	For countable sets we will sometimes view permutations and total orders as sequences of elements using a subscript notation as long as our meaning is clear from the context.

	\begin{definition}
		Throughout this paper we will use $n$ to represent the number of voters in an election, and $m$ to represent the number of candidates. Let $C = \{1, \ldots, m\}$ be the set of all \emph{alternatives} (candidates). We define the set of all \emph{preference lists} to be $V = L(C)$. We can also view a preference list as a permutation on $C$; it will be obvious from context which approach we a using. We define the set of all \emph{preference profiles} to be $P = L(C)^n$. We define a \emph{voting rule}, or \emph{social choice function} (SCF), to be a function $f : P \to C$. And lastly we define an \emph{election} to be simply a voting rule paired with a profile: $(f, p)$ where $f$ is a voting rule and $p \in P$.
	\end{definition}

	\begin{definition}
		A \emph{successful manipulation} (or \emph{profitable manipulation}) by voter $i$ of a SCF $f$ at profile $x$ is a preference $x'_i$ such that
		\[
			f((x_{-i}, x'_i)) \succ_i f((x_{-i}, x_i))
		\]
	\end{definition}

	\begin{definition}
		Let $v \in V$ be a preference list, and let $x, y \in C$ be two alternatives. Since $v$ is actually a total ordering, we denote $(x, y) \in v$ by
		\[
			x <_v y
		\]
		and if this is the case we view $x$ as being ranked above $y$ in $v$ and we say that $x$ beats $y$, and denote this as
		\[
			x \succ_v y
		\]
		We view $x$ as being ranked below $y$ in $v$ if
		\[
			x >_v y
		\]
		and we would say that $x$ is beaten by $y$, we denote this as
		\[
			x \prec_v y
		\]
	\end{definition}

	\begin{definition}
		For a set of candidates $D \subseteq C$, for a preference list $v \in V$ and a preference profile $p \in P$ we denote $v$ and $p$ \emph{restricted to} $D$ by $v|_D$ and $p|_D$ respectively. $v|_D$ means $v$ after all the candidates who are not in $D$ have been removed from the preference list. $p|_D$ means that every preference list in $p$ has been restricted to $D$. For any sequence $v$, and $i \in \{1, \ldots, |v|\}$ we will denote by $v_{-i}$, $v$ with $v_i$ removed.
	\end{definition}

	\begin{definition}
		We define a \emph{poset}, or \emph{partially ordered set}, to be $(X, \le)$ where $X$ is a set, and $\le$ is a binary relation on $X$. $\le$ is also called a partial ordering because of the fact that not every pair of elements in $X$ needs to be related by $\le$, as opposed to a total ordering which must relate every pair.
	\end{definition}

	\begin{definition}
		For any poset $(P, \le)$, an \emph{upper bound} of a subset $X \subseteq P$ is an element $a \in P$ such that $a \le x$ for every $x \in X$. A \emph{least upper bound} is an \emph{upper bound} that is less than or equal to every other \emph{upper bound}. We denote this \emph{least upper bound} as $\sup_P X$ calling it the \emph{supremum} \cite{birkhoff1967lattice} and also as $\bigjoin_P X$ calling it the \emph{join}. When $X$ contains only two elements, we can use the join as a binary operator: $\bigjoin_P \{a, b\} = a \join_P b$. When $P$ is obvious from context we will simply write $\sup X$ or $\bigjoin X$. If the supremum exists, it is unique because posets are antisymmetric. The supremum is the same as the infimum in the inverse order.
	\end{definition}

	\begin{definition}
		For any poset $(P, \le)$, a \emph{lower bound} of a subset $X \subseteq P$ is an element $a \in P$ such that $a \ge x$ for every $x \in X$. A \emph{greatest lower bound} is a \emph{lower bound} that is greater than or equal to every other \emph{lower bound}. We denote this \emph{greatest lower bound} as $\inf_P X$ calling it the \emph{infimum} \cite{birkhoff1967lattice} and also as $\bigmeet_P X$ calling it the \emph{meet}. When $X$ contains only two elements, we can use the meet as a binary operator: $\bigmeet_P \{a, b\} = a \meet_P b$. When $P$ is obvious from context we will simply write $\inf X$ or $\bigmeet X$. If the infimum exists, it is unique because posets are antisymmetric. The infimum is the same as the supremum in the inverse order.
	\end{definition}

	\begin{definition}
		A poset, $(P, \le)$, is a \emph{lattice} if for any $x, y \in P$ the meet and join of $x$ and $y$ both exist. Note that the meet and join are unique by definition (if they exist).
	\end{definition}

	\begin{definition}
		A binary relation $R$ on a set $X$ is \emph{transitive relation} if $\forall a,b,c \in X$
		\[
			(aRb \textrm{ and } bRc) \implies aRc
		\]
	\end{definition}

	\begin{definition}
		The \emph{transitive closure} of a binary relation $R$ on a set $X$ is the transitive relation $R^t$ on $X$ such that $R \subseteq R^t$ and $R^t$ is minimal \cite[p. 337]{lidl1998applied}.
	\end{definition}

	\begin{definition}
		For any poset $(P, \le)$, let $\sigma$ be a permutation of $P$. We define the \emph{inversions} of $\sigma$ to be a binary relation $\Inv_{\sigma}$ on $P$:
		\[
			\Inv_{\sigma} = \{(i,j) \mid i, j \in P, i < j, \sigma^{-1}(i) > \sigma^{-1}(j)\}
		\]
		We can read $i \Inv_{\sigma} j$ as ``$i$ is inverted with $j$ in $\sigma$''. $\Inv$ is a transitive relation because for any $i,j,k \in P$ if $i \Inv_{\sigma} j$ and $j \Inv_{\sigma} k$ then $i < j < k$ and $\sigma^{-1}(i) > \sigma^{-1}(k) > \sigma^{-1}(k)$ which means that $i \Inv_{\sigma} k$.

		In addition, let $(X, \le')$ be a lattice such that the elements of $X$ are permutations of $P$. For any $\sigma, \pi \in X$ we have
		\[
			\Inv_{\sigma \meet \pi} = (\Inv_{\sigma} \cup \Inv_{\pi})^t
		\]
		\cite{markowsky1994permutation}.
	\end{definition}

	\begin{definition}
		For any poset $(P, \le)$, let $x,y \in P$. We say that $x$ is a \emph{predecessor} of $y$ if $x < y$. We say that $x$ is a \emph{direct predecessor} of $y$ if $x$ is the greatest predecessor of $y$.
	\end{definition}

	\begin{definition}
		For any poset $(P, \le)$, let $x,y \in P$. We say that $x$ is a \emph{successor} of $y$ if $x > y$. We say that $x$ is a \emph{direct successor} of $y$ if $x$ is the least successor of $y$.
	\end{definition}

	\begin{definition}[$X^{ij}, \le^{ij}$]
		Let $(X, \le)$ be a lattice whose elements are permutations of a set $Y$. For any $i,j \in Y$ we define
		\[
			X^{ij} = \{ x \in X \mid x^{-1}(i) < x^{-1}(j) \}
		\]
		We then define the partial ordering, $\le^{ij}$, over $X^{ij}$ such that for $x, y \in X^{ij}$:
		\[
			x \le^{ij} y \iff x \le y
		\]
	\end{definition}

	\begin{definition}[$\le_s$]
		Let $(P, \le)$ be a poset and let $X$ be the set of all permutations on $P$. We define the partial ordering $\le_s$ on $X$ such that for all $\sigma, \pi \in X$:
		\[
			\sigma \le_s \pi \iff \Inv_{\sigma} \subseteq \Inv_{\pi}
		\]
	\end{definition}

	\begin{lemma}
		\label{identified-permutation-lattice-join}
		Let $(X, \le_s)$ be a lattice whose elements are permutations of a set $Y$, and $\le_s$ is defined as above. Let $\Inv$ be the inversion binary relation over $Y$ as defined above. Let $\join$ and $\join^{ij}$ denote the join in $(X, \le_s)$ and $(X^{ij}, \le^{ij}_s)$ respectively. For any $i,j \in Y$, if $i$ is either a direct successor or a direct predecessor of $j$ according to $\le_s$, then for all $x, y \in X^{ij}$:
		\[
			\exists(x \join y) \implies \exists(x \join^{ij} y)
		\]
	\end{lemma}

	\begin{proof}
		Assume $\exists(x \join y)$. Let $z = x \join y$. Then $z$ is an upper bound of $\{x, y\}$:
		\[
			z \ge_s x \textrm{ and } z \ge_s y
		\]
		And $z$ is the least upper bound of $\{x, y\}$: for every $a \in X$:
		\[
			(a \ge_s x \textrm{ and } a \ge_s y) \implies z \le_s a
		\]
		Since $x \in X^{ij}$, then $(i, j) \notin Inv_x$. Since $z \ge_s x$, then $(i, j) \notin Inv_z$, so $z \in X^{ij}$. By definition $z \ge_s x \implies z \ge^{ij}_s x$ and $z \ge_s y \implies z \ge^{ij}_s y$. Therefore $z$ is an upper bound of $\{x, y\}$ in $X^{ij}$.

		For any $a \in X^{ij}$ if $a$ is an upper bound of $\{x, y\}$ in $X^{ij}$ then clearly $a$ is also an upper bound of $\{x, y\}$ in $X$. Therefore $z \le_s a$, so $z \le^{ij}_s a$, which means $z = x \join^{ij} y$. So clearly $x \join^{ij} y$ exists.
	\end{proof}

	\begin{lemma}
		\label{identified-permutation-lattice-meet}
		Let $(X, \le_s)$ be a lattice whose elements are permutations of a set $Y$, and $\le_s$ is defined as above. Let $\Inv$ be the inversion binary relation over $Y$ as defined above. Let $\meet$ and $\meet^{ij}$ denote the join in $(X, \le_s)$ and $(X^{ij}, \le^{ij}_s)$ respectively. For any $i,j \in Y$, if $i$ is either a direct successor or a direct predecessor of $j$ according to $\le_s$, then for all $x, y \in X^{ij}$:
		\[
			\exists(x \meet y) \implies \exists(x \meet^{ij} y)
		\]
	\end{lemma}

	\begin{proof}
		Assume $\exists(x \meet y)$. Let $z = x \meet y$. Then $z$ is a lower bound of $\{x, y\}$:
		\[
			z \le_s x \textrm{ and } z \le_s y
		\]
		And $z$ is the greatest lower bound of $\{x, y\}$: for every $a \in X$:
		\[
			(a \le_s x \textrm{ and } a \le_s y) \implies z \ge_s a
		\]

		We will now detour to show that $z \in X^{ij}$. Since $z = x \meet y$, then $\Inv_z = (\Inv_x \cup \Inv_y)^t$ \cite{markowsky1994permutation}. Because $x,y \in X^{ij}$ we know that $(i, j) \notin (\Inv_x \cup \Inv_y)$. Therefore, in order to have $(i, j) \in (\Inv_x \cup \Inv_y)^t$ we would need to have $(i, k) \in \Inv_x$ and $(k, j) \in \Inv_y$ for any $k \in Y$, which is impossible because $i$ is either a direct successor or a direct predecessor of $j$. Therefore $(i, j) \notin Inv_z$, so $z \in X^{ij}$.

		By definition $z \le_s x \implies z \le^{ij}_s x$ and $z \le_s y \implies z \le^{ij}_s y$. Therefore $z$ is a lower bound of $\{x, y\}$ in $X^{ij}$.

		For any $a \in X^{ij}$ if $a$ is a lower bound of $\{x, y\}$ in $X^{ij}$ then clearly $a$ is also a lower bound of $\{x, y\}$ in $X$. Therefore $z \ge_s a$, so $z \ge^{ij}_s a$, which means $z = x \meet^{ij} y$. So clearly $x \meet^{ij} y$ exists.
	\end{proof}

	\begin{proposition}
		\label{proposition-identification-is-lattice}
		Let $(X, \le_s)$ be a lattice whose elements are permutations of a set $Y$, and $\le_s$ is defined as above. Let $\Inv$ be the inversion binary relation over $Y$ as defined above. For any $i,j \in Y$, if $i$ is either a direct successor or a direct predecessor of $j$ according to $\le_s$, then $(X^{ij}, \le^{ij}_s)$ is a lattice.
	\end{proposition}

	\begin{proof}
		We know that $\exists(x \join y)$ and $\exists(x \meet y)$ because $(X, \le_s)$ is a lattice. Therefore by lemma \ref{identified-permutation-lattice-join} and lemma \ref{identified-permutation-lattice-meet} we have $\exists(x \join^{ij} y)$ and $\exists(x \meet^{ij} y)$ respectively. So $(X^{ij}, \le^{ij}_s)$ is a lattice, by definition of a lattice.
	\end{proof}

	\begin{proposition}
		\label{proposition-grid-is-lattice}
		Let $(X, \le)$ be a lattice. Let $X^n$ be the set of all $n$-tuples of elements of $X$. Let $\le^n$ be defined as: for all $x, y \in X$ and all $i \in \{1, \ldots, n\}$
		\[
			x \le^n y \iff x_i \le y_i
		\]
		$(X^n, \le^n)$ is a lattice.
	\end{proposition}

	\begin{proof}
		By definition of a lattice, $(X^n, \le^n)$ is a lattice if for any two elements $s, t \in S^n$, $s \join t$ exists and $s \meet t$ exists.

		First we show that $s \join t$ exists. We define $u \in X^n$ such that $u_i = s_i \join t_i$, $\forall i \in \{1, \ldots, n\}$, and we show that $u = s \join t$. Because $u_i = s_i \join t_i$, we have
		\[
			u_i \ge s_i \textrm{ and } u_i \ge s_i
		\]
		so
		\[
			u \ge^n s \textrm{ and } u \ge^n t
		\]
		meaning that $u$ is an upper bound for $s$ and $t$. Suppose there is some $v \in X^n$ which is also an upper bound for $s$ and $t$. Then $\forall i \in \{1, \ldots, n\}$ we have
		\[
			v_i \ge s_i \textrm{ and } v_i \ge t_i
		\]
		so since $u_i = s_i \join t_i$, then $u_i \le v_i$. Therefore $u \le^n v$, i.e. $u$ is the least upper bound of $\{s, t\}$.

		Second we show that $s \meet t$ exists (by the same argument). We define $u \in X^n$ such that $u_i = s_i \meet t_i$, $\forall i \in \{1, \ldots, n\}$, and we show that $u = s \meet t$. Because $u_i = s_i \meet t_i$, we have
		\[
			u_i \le s_i \textrm{ and } u_i \le s_i
		\]
		so
		\[
			u \le^n s \textrm{ and } u \le^n t
		\]
		meaning that $u$ is a lower bound for $s$ and $t$. Suppose there is some $v \in X^n$ which is also a lower bound for $s$ and $t$. Then $\forall i \in \{1, \ldots, n\}$ we have
		\[
			v_i \le s_i \textrm{ and } v_i \le t_i
		\]
		so since $u_i = s_i \meet t_i$, then $u_i \ge v_i$. Therefore $u \ge^n v$, i.e. $u$ is the greatest lower bound of $\{s, t\}$.
	\end{proof}

%!TEX root = thesis.tex

\chapter{Step Three of Friedgut}

	Friedgut's main theorem is proved in three steps; the first two are generalized. Therefore to generalize the main theorem we need only generalize the third step. This step is comprised of lemma 6, lemma 7, and lemma 8 which we will generalize one at a time.

	\begin{lemma}[Lemma 3 of Friedgut]
		For every SCF $f$ on $m$ alternatives and every $a, b \in C$:
		\[
			M^{a, b}(f) \le m! \cdot \sum_i M_i(f)
		\]
	\end{lemma}

	For the rest of the proof we will fix a SCF $f$.


\section{Generalized Lemma 6 of Friedgut}

	For any preference profile $p \in P$ there are $(\frac{m!}{2})^n$ profiles $x$ such that $x|_{\{a, b\}} = p|_{\{a, b\}}$. This is because there are $m!$ possible preference lists; half of them will have the preference between $a$ and $b$ that agrees with $p|_{\{a, b\}}$ and half will disagree. For each voter this gives $\frac{m!}{2}$ possible preference lists which gives $(\frac{m!}{2})^n$ profiles comprised of these preference lists.

	\begin{definition}
		Let $C$ be a set of alternatives. Let $a, b, c \in C$ be any three alternatives. Let $p \in L(C)^n$ be a preference profile for $n$ voters, and let $f$ be a SCF. We define
		\[
			A^{a,b}_c(p) = \{x \in L(C)^n \mid x|_{\{a,b\}} = p|_{\{a,b\}}, f(x) = c\}
		\]
	\end{definition}

	Therefore we can rewrite $M^{a,b}(f)$ as follows.

	\begin{lemma}[Lemma 6 of Friedgut] Let $C$ be a set of alternatives. Let $a, b \in C$ be any two alternatives. Let $m = |C|$ and let $n$ be the number of voters. Let $f$ be a SCF. We have
		\[
			M^{a,b}(f) = E_{p \in L(C)^n} \left[ \frac{|A^{a,b}_a(p)|}{\left(\frac{m!}{2}\right)^n} \cdot \frac{|A^{a,b}_b(p)|}{\left(\frac{m!}{2}\right)^n} \right]
		\]
	\end{lemma}

	\begin{proof}
		This is just a rewording of the definition of $M^{a,b}(f)$.
	\end{proof}


\section{Generalized Lemma 7 of Friedgut}

	We now attempt to relate $M_i(f)$ to $A$.

	Let $n$ be the number of voters. Let $C = \{1, \ldots, m\}$ be a set of alternatives, and let $a,b \in C$ be any two alternatives. We define an anonymized version of the alternatives as $C' = \{c_1, \ldots, c_m\}$, totally ordered as $c_1 < \ldots < c_m$.

	\begin{definition}
		$C'$ is isomorphic to $C$ and we define the mapping function $g^{a,b} : C \to C'$ such that
		\begin{align*}
			g^{a,b}(x) =
			\begin{cases}
				c_x & \textrm{if } x \in C \backslash \{1, 2, a, b\} \\
				c_1 & \textrm{if } x = a \\
				c_2 & \textrm{if } x = b \\
				c_a & \textrm{if } x = 1 \\
				c_b & \textrm{if } x = 2
			\end{cases}
		\end{align*}
		We define $G^{a,b} : L(C) \to L(C')$ such that
		\[
			G^{a,b}(x) = (g^{a,b}(x_1), \ldots, g^{a,b}(x_m))
		\]
	\end{definition}

	\begin{definition}
		We define the partial ordering, $\le^G_s$, on $L(C)$ such that for all $x, y \in L(C)$:
		\[
			x \le^g_s y \iff g^{a,b}(x) \le_s g^{a,b}(y)
		\]
	\end{definition}

	Clearly $(L(C), \le^G_s)$ is a lattice.

	\begin{definition}
		We define the partial order $(\le^G_s)^n$ on $L(C)^n$ such that for all $x,y \in L(C)^n$ and all $i \in \{1, \ldots, n\}$:
		\[
			x (\le^G_s)^n y \iff x_i \le^G_s y_i
		\]
	\end{definition}

	Clearly $(L(C)^n, (\le^G_s)^n)$ is a lattice.

	\begin{definition}
		Let $p \in L(C)^n$. We define the \emph{upper edge border} of $A^{a,b}_a(p)$, denoted $\partial A^{a,b}_a(p)$, to be the set of directed edges whose tail is in $A^{a,b}_a(p)$ and whose head is not in $A^{a,b}_a(p)$. Formally, for all $i \in \{1, \ldots, n\}$:
			\[
				\partial_i A^{a,b}_a(p) = \{ (x_{-i}, x_i, x'_i) \mid (x_{-i}, x_i) \in A^{a,b}_a(p), (x_{-i}, x'_i) \notin A^{a,b}_a(p), x_i <^G_s x'_i \}
			\]
		and
			\[
				\partial A^{a,b}_a(p) = \bigcup_j \partial_j A^{a,b}_a(p)
			\]
		We define the upper edge border of $A^{a,b}_b(p)$ analogously.
	\end{definition}

	\begin{lemma}
		\label{manipulation-per-edge-in-a}
		Let $p, p' \in L(C)^n$ be profiles such that for all $i \in \{1, \ldots, n\}$:
		\begin{align*}
			p_{-i} &= p'_{-i} \textrm{ and} \\
			p_i|_{a,b} &= p'_i|_{a,b}
		\end{align*}
		If either
		\begin{align*}
			(x_{-i}, x_i, x'_i) &\in \partial_i A^{a,b}_a(p) \textrm{ or} \\
			(x_{-i}, x_i, x'_i) &\in \partial_i A^{a,b}_b(p)
		\end{align*}
		then the pair $p, p'$ corresponds to at least one successful manipulation.
	\end{lemma}

	\begin{proof}
		By definition of the upper edge border we have
		\[
			x_i \le^G_s x'_i
		\]
		And by definition of $A^{a,b}_a$ and $A^{a,b}_b$ we have
		\[
			x_i|_{\{a,b\}} = x'_i|_{\{a,b\}}
		\]

		For $(x_{-i}, x_i, x'_i) \in \partial_i A^{a,b}_a(p)$, we know that $f((x_{-i}, x_i)) = a$ and $f((x_{-i}, x'_i)) = t$ for $t \in C \backslash \{a\}$. If $t \succ_{x_i} a$ then $x'_i$ is a successful manipulation of $(x_{-i}, x_i)$. Otherwise, $a \succ_{x_i} t$. If this is the case, then we know that $(a, t) \notin Inv_{x_i}$, and because $x_i \le^G_s x'_i$ we have $(a, t) \notin Inv_{x'_i}$, which means $a \succ_{x'_i} t$. Therefore $x_i$ is a successful manipulation of $(x_{-i}, x'_i)$.

		And analogously for $(x_{-i}, x_i, x'_i) \in \partial_i A^{a,b}_b(p)$, either $x'_i$ is a successful manipulation of $(x_{-i}, x_i)$ or $x_i$ is a successful manipulation of $(x_{-i}, x'_i)$.
	\end{proof}

	\begin{lemma}[Lemma 7 of Friedgut]
		\[
			M_i(f) \ge \frac{1}{m!} \left(\frac{m!}{2}\right)^{-n} E_x \left[|\partial_i A^{a,b}_a(p)| + |\partial_i A^{a,b}_b(p)| \right]
		\]
	\end{lemma}

	\begin{proof}
		Recall the definition of $M_i(f)$: given a profile $p \in P$ and vote $p'_i \in V$ chosen uniformly at random, $M_i(f)$ is the probability that $p'_i$ is a successful manipulation of $p$ by voter $i$. Therefore to lower bound $M_i(f)$ we start with $p$ and $p'_i$ chosen uniformly at random. We can think of these as two distinct profiles, $p$ and $p'$, where $p' = (p_{-i}, p'_i)$.

		Clearly $p_{-i}|_{\{a,b\}} = p'_{-i}|_{\{a,b\}}$, but we will have $p_i|_{\{a,b\}} = p'_i|_{\{a,b\}}$ only with probability $\frac{1}{2}$, and we condition the following on this being the case. So we have $p|_{\{a,b\}} = p'|_{\{a,b\}}$.

		By lemma \ref{manipulation-per-edge-in-a}, each $(x_{-i}, x_i, x'_i) \in (\partial_i A^{a,b}_a(p) \cup \partial_i A^{a,b}_b(p))$ corresponds to at least one successful manipulation. Note that if $(x_{-i}, x_i, x'_i) \in \partial_i A^{a,b}_a(p)$ then $(x_{-i}, x'_i, x_i) \notin \partial_i A^{a,b}_a(p)$.

		Therefore we can lower bound $M_i(f)$ by the probability that an edge is in either $\partial_i A^{a,b}_a(p)$ or $\partial_i A^{a,b}_b(p)$. The total possible number of edges is
		\[
			\frac{m!}{2} \cdot \frac{m!}{2} \cdot \left(\frac{m!}{2}\right)^{n-1} = \frac{m!}{2}\left(\frac{m!}{2}\right)^{n}
		\]
		So the probability that a randomly chosen edge is in either $\partial_i A^{a,b}_a(p)$ or $\partial_i A^{a,b}_b(p)$ is
		\[
			\frac{2}{m!} \left(\frac{2}{m!}\right)^{n} \cdot E \left[ |\partial_i A^{a,b}_a(p)| + |\partial_i A^{a,b}_b(p)| \right]
		\]
		Note that we can sum the probabilities for $\partial_i A^{a,b}_a(p)$ and $\partial_i A^{a,b}_b(p)$ because they are disjoint by the definition of the upper edge border; an edge cannot satisfy both $(x_{-i}, x_i) \in A^{a,b}_a(p)$ and $(x_{-i}, x_i) \in A^{a,b}_b(p)$ simultaneously because if $f((x_{-i}, x_i)) = a$ then $f((x_{-i}, x_i)) \ne b$ and vice versa.

		We conditioned our analysis on $p_i = p'_i$, so then, our lower bound becomes
		\[
			M_i(f) \ge \frac{1}{2} \cdot \frac{2}{m!}\left(\frac{2}{m!}\right)^{n} \cdot E \left[ |\partial_i A^{a,b}_a(p)| + |\partial_i A^{a,b}_b(p)| \right]
		\]
		And simplified
		\[
			M_i(f) \ge \frac{1}{m!}\left(\frac{2}{m!}\right)^{n} \cdot E \left[ |\partial_i A^{a,b}_a(p)| + |\partial_i A^{a,b}_b(p)| \right]
		\]
	\end{proof}

	Summing over $i$ we get

	\begin{corollary}[Corollary 1 of Friedgut]
		\[
			\frac{1}{m!} \cdot \left(\frac{m!}{2}\right)^{-n} E_p[|\partial A^{a,b}_a(p)| + |\partial A^{a,b}_b(p)|] \le \sum_i M_i(f)
		\]
	\end{corollary}


\section{Generalized Lemma 8 of Friedgut}

	In this section we will fix candidates $a, b$ and profile $p$, and for the sake of readability we will define the following sets:
	\begin{align*}
		A &= A^{a,b}_a(p) \\
		B &= A^{a,b}_b(p)
	\end{align*}

	\begin{lemma}
		For every disjoint $A, B$ we have that
		\[
			|\partial A| + |\partial B| \ge \left( \frac{2}{m!} \right)^n |A| \cdot |B|
		\]
	\end{lemma}

	\begin{proof}
		We first define a total ordering, $\le_i^t$, over each dimension of $P$ as follows. Let $x, y \in P$ and let $i \in \{1, \ldots, n\}$.
		\[
			x_i \le^G_s y_i \implies x_i \le_i^t y_i
		\]
		Otherwise, sort the pairs of $\Inv_{x_i} \cup \Inv_{y_i}$ primarily by the first item of the pair, and secondarily by the second item. Iterate through this sorted list and when an item, $z_i$, is reached such that
		\[
			z_i \in \Inv_{x_i} \oplus \Inv_{y_i}
		\]
		then
		\begin{align*}
			z_i &\in \Inv_{x_i} \implies x_i \le_i^t y_i \\
			z_i &\in \Inv_{y_i} \implies y_i \le_i^t x_i
		\end{align*}
		Note: we can sort $\Inv_{x_i} \cup \Inv_{y_i}$ because $C$ is totally ordered.

		We claim that $\le_i^t$ is now a total ordering over $V$.

		We define $A'$ to be a consolidation of $A$ as follows. We start with $A' = A$. We iterate over $i \in \{1, \ldots, n\}$. For each $x \in A'$, for each $j \in \{1, \ldots, n\} \backslash \{i\}$ we replace $x$ with $y \in P \backslash A'$ such that the following holds
		\begin{align*}
			y_i &\le_i^t x_i \\
			y_j &= x_j
		\end{align*}
		if such a $y$ exists.




		\[
			y \le_t x
		\]
		and $\forall z \in P \backslash A' \cup y$
		\[
			z \le_t x \implies y \le_t z
		\]
		In other words, we add the least profile not already in $A'$ which is less than or equal to $x$. In a visual sense, we are consolidating the elements of $A$ towards the bottom of the ordering $\le_t$.

		It is easy to check that
		\begin{align*}
			|\partial A'| &\le |\partial A| \textrm{ and} \\
			|\partial B'| &\le |\partial B|
		\end{align*}

		We will now show that if a profile is in $|A' \backslash A|$ then it is also in $\partial A$. Let $y \in |A' \backslash A|$. Since $y \in A'$ we know that there is a corresponding $x \in A$ such that
		\[
			y (\le^G_s)^n x
		\]
		But because $y \notin A$ we know that
		\[
			y (\ne^G_s)^n x
		\]
		So we have that $\exists i \in \{1, \ldots, n\}$ such that
		\[
			y_i <^G_s x_i
		\]
		which means that the edge $(x_{-i}, x_i, y_i) \in \partial_i A$ so $(x_{-i}, x_i, y_i) \in \partial A$ as well.

		Since every profile in $|A' \backslash A|$ corresponds to at least one profile in $\partial A$ we know that
		\[
			|A' \backslash A| \le |\partial A|
		\]
		and likewise for $B'$:
		\[
			|B' \backslash B| \le |\partial B|
		\]

		Since for any two votes $v_1, v_2 \in A \cup B$ we have $v_1|_{\{a,b\}} = v_2|_{\{a,b\}}$ we can define a new set
		\[
			P' = \{x \in P \mid x|_{\{a,b\}} = p|_{\{a,b\}}\}
		\]
		and view $A$, $B$, $A'$, and $B'$ as residing in $P'$ without losing any information. Clearly $|P'| = (\frac{m!}{2})^n$.

		For any vote $v \in P'$, let $E_{A'}$ be the event that $v$ is in $A'$, and let $E_{B'}$ be the event that $v$ is in $B'$. Then
		\[
			P(E_{A'} \cap E_{B'}) = P(E_{A'}) P(E_{B'}|E_{A'})
		\]
		Clearly
		\begin{align}
			\label{probability-values-1}
			P(E_{A'} \cap E_{B'}) &= \frac{|A' \cap B'|}{|P'|} \\
			\label{probability-values-2}
			P(E_{A'}) &= \frac{|A'|}{|P'|} \\
			\label{probability-values-3}
			P(E_{B'}) &= \frac{|B'|}{|P'|}
		\end{align}
		We claim that $A'$ and $B'$ are positively correlated, and so knowing that an element is in $A'$ only increases the chances that it is in $B'$. In other words
		\[
			P(E_{B'}|E_{A'}) \ge P(E_{B'})
		\]
		So we have
		\[
			P(E_A \cap E_B) \ge P(E_A) P(E_B)
		\]
		So by substitution from equations \ref{probability-values-1}, \ref{probability-values-2}, and \ref{probability-values-3} we get
		\begin{align*}
			\frac{|A' \cap B'|}{(\frac{m!}{2})^n} &\ge \frac{|A'|}{(\frac{m!}{2})^n} \frac{|B'|}{(\frac{m!}{2})^n} \\
			&= \frac{|A|}{(\frac{m!}{2})^n} \frac{|B|}{(\frac{m!}{2})^n}
		\end{align*}
		However $A$ and $B$ are disjoint so
		\[
			A' \cap B' \subseteq (A' \backslash A) \cup (B' \backslash B)
		\]
		which completes the proof as follows
		\begin{align*}
			|A' \cap B'| &\le |A' \backslash A| + |B' \backslash B| \\
			|A' \cap B'| &\le |\partial A| + |\partial B| \\
			\frac{|A||B|}{(\frac{m!}{2})^n} &\le |\partial A| + |\partial B| \\
			\left(\frac{2}{m!}\right)^n |A| \cdot |B| &\le |\partial A| + |\partial B| \\
			|\partial A| + |\partial B| &\ge \left(\frac{2}{m!}\right)^n |A| \cdot |B| \\
		\end{align*}
	\end{proof}

\section{Finished Step 3 of Friedgut}

	Lemma 6, 7, and 8 fit together as follows. First we define the variables $L_6$, $L_7$, and $L_8$ to be variable values that multiply each of the lemmas respectively. The values of these variables will change depending on the value of $m$, so we evaluate the lemmas in terms of these variables to be more general. We can define the lemmas in terms of these variables:
	\begin{align*}
		&M^{a,b} = E[|A||B|] \cdot L_6 & \textrm{lemma 6} \\
		&L_7 \cdot E[|\partial A| + |\partial B|] \le \sum_i M_i & \textrm{lemma 7} \\
		&\frac{1}{L_8} \cdot (|\partial A| + |\partial B|) \ge |A||B| & \textrm{lemma 8}
	\end{align*}

	Now we can solve for the result of step 3.
	\begin{align*}
		M^{a,b} &= E[|A||B|] \cdot L_6 & \textrm{by lemma 6} \\
		M^{a,b} &\le E[|\partial A| + |\partial B|] \cdot \frac{L_6}{L_8} & \textrm{by lemma 8} \\
		M^{a,b} &\le \sum_i M_i \cdot \frac{L_6}{L_7L_8} & \textrm{by lemma 7}
	\end{align*}

	If we can fully generalize this step and capture all of the $v_i$'s our results will, possibly, look like this:
	\begin{align*}
		L_6 &= \left(\frac{m!}{2}\right)^{-2n} \\
		L_7 &= \frac{1}{m!}\left(\frac{m!}{2}\right)^{-n} \\
		L_8 &= \left(\frac{m!}{2}\right)^{-n}
	\end{align*}

	So we have that
	\begin{align*}
		\frac{L_6}{L_7L_8} &= \left(\frac{m!}{2}\right)^{-2n} \cdot m!\left(\frac{m!}{2}\right)^{n} \cdot \left(\frac{m!}{2}\right)^{n} \\
		&= \left(\frac{m!}{2}\right)^{-2n} \cdot m! \cdot \left(\frac{m!}{2}\right)^{2n} \\
		&= m!
	\end{align*}

	And the final result for step 3 becomes
	\begin{align*}
		M^{a,b} &\le \sum_i M_i \cdot m!
	\end{align*}



\backmatter

% Bibliography
\bibliographystyle{abbrv}
\bibliography{../../Common/references}  % file path without .bib extension

\end{document}
