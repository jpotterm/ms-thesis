%!TEX root = thesis.tex

\chapter{Results}

	Friedgut's main theorem is proved in three steps; the first two are generalized. Therefore to generalize the main theorem we need only generalize the third step. This step is comprised of lemma 6, lemma 7, and lemma 8 which we will generalize one at a time.

	\begin{lemma}[Lemma 3 of Friedgut]
		For every SCF $f$ on $m$ alternatives and every $a, b \in C$:
		\[
			M^{a, b}(f) \le m! \cdot \sum_i M_i(f)
		\]
	\end{lemma}

	For the rest of the proof we will fix a SCF $f$.


\section{Generalized Lemma 6 of Friedgut}

	For any preference profile $p \in P$ there are $(\frac{m!}{2})^n$ profiles $x$ such that $x|_{\{a, b\}} = p|_{\{a, b\}}$. This is because there are $m!$ possible preference lists; half of them will have the preference between $a$ and $b$ that agrees with $p|_{\{a, b\}}$ and half will disagree. For each voter this gives $\frac{m!}{2}$ possible preference lists which gives $(\frac{m!}{2})^n$ profiles comprised of these preference lists.

	\begin{definition}
		Let $C$ be a set of alternatives. Let $a, b, c \in C$ be any three alternatives. Let $p \in L(C)^n$ be a preference profile for $n$ voters, and let $f$ be a SCF. We define
		\[
			A^{a,b}_c(p) = \{x \in L(C)^n \mid x|_{\{a,b\}} = p|_{\{a,b\}}, f(x) = c\}
		\]
	\end{definition}

	Therefore we can rewrite $M^{a,b}(f)$ as follows.

	\begin{lemma}[Lemma 6 of Friedgut] Let $C$ be a set of alternatives. Let $a, b \in C$ be any two alternatives. Let $m = |C|$ and let $n$ be the number of voters. Let $f$ be a SCF. We have
		\[
			M^{a,b}(f) = E_{p \in L(C)^n} \left[ \frac{|A^{a,b}_a(p)|}{\left(\frac{m!}{2}\right)^n} \cdot \frac{|A^{a,b}_b(p)|}{\left(\frac{m!}{2}\right)^n} \right]
		\]
	\end{lemma}

	\begin{proof}
		This is just a rewording of the definition of $M^{a,b}(f)$.
	\end{proof}


\section{Generalized Lemma 7 of Friedgut}

	We now attempt to relate $M_i(f)$ to $A$.

	Let $n$ be the number of voters. Let $C = \{1, \ldots, m\}$ be a set of alternatives, and let $a,b \in C$ be any two alternatives. We define an anonymized version of the alternatives as $C' = \{c_1, \ldots, c_m\}$, totally ordered as $c_1 < \ldots < c_m$.

	\begin{definition}
		$C'$ is isomorphic to $C$ and we define the mapping function $g^{a,b} : C \to C'$ such that
		\begin{align*}
			g^{a,b}(x) =
			\begin{cases}
				c_x & \textrm{if } x \in C \backslash \{1, 2, a, b\} \\
				c_1 & \textrm{if } x = a \\
				c_2 & \textrm{if } x = b \\
				c_a & \textrm{if } x = 1 \\
				c_b & \textrm{if } x = 2
			\end{cases}
		\end{align*}
		We define $G^{a,b} : L(C) \to L(C')$ such that
		\[
			G^{a,b}(x) = (g^{a,b}(x_1), \ldots, g^{a,b}(x_m))
		\]
	\end{definition}

	\begin{definition}
		We define the partial ordering, $\le^G_s$, on $L(C)$ such that for all $x, y \in L(C)$:
		\[
			x \le^g_s y \iff g^{a,b}(x) \le_s g^{a,b}(y)
		\]
	\end{definition}

	Clearly $(L(C), \le^G_s)$ is a lattice.

	\begin{definition}
		We define the partial order $(\le^G_s)^n$ on $L(C)^n$ such that for all $x,y \in L(C)^n$ and all $i \in \{1, \ldots, n\}$:
		\[
			x (\le^G_s)^n y \iff x_i \le^G_s y_i
		\]
	\end{definition}

	Clearly $(L(C)^n, (\le^G_s)^n)$ is a lattice.

	\begin{definition}
		Let $p \in L(C)^n$. We define the \emph{upper edge border} of $A^{a,b}_a(p)$, denoted $\partial A^{a,b}_a(p)$, to be the set of directed edges whose tail is in $A^{a,b}_a(p)$ and whose head is not in $A^{a,b}_a(p)$. Formally, for all $i \in \{1, \ldots, n\}$:
			\[
				\partial_i A^{a,b}_a(p) = \{ (x_{-i}, x_i, x'_i) \mid (x_{-i}, x_i) \in A^{a,b}_a(p), (x_{-i}, x'_i) \notin A^{a,b}_a(p), x_i <^G_s x'_i \}
			\]
		and
			\[
				\partial A^{a,b}_a(p) = \bigcup_j \partial_j A^{a,b}_a(p)
			\]
		We define the upper edge border of $A^{a,b}_b(p)$ analogously.
	\end{definition}

	\begin{lemma}
		\label{manipulation-per-edge-in-a}
		Let $p, p' \in L(C)^n$ be profiles such that for all $i \in \{1, \ldots, n\}$:
		\begin{align*}
			p_{-i} &= p'_{-i} \textrm{ and} \\
			p_i|_{a,b} &= p'_i|_{a,b}
		\end{align*}
		If either
		\begin{align*}
			(x_{-i}, x_i, x'_i) &\in \partial_i A^{a,b}_a(p) \textrm{ or} \\
			(x_{-i}, x_i, x'_i) &\in \partial_i A^{a,b}_b(p)
		\end{align*}
		then the pair $p, p'$ corresponds to at least one successful manipulation.
	\end{lemma}

	\begin{proof}
		By definition of the upper edge border we have
		\[
			x_i \le^G_s x'_i
		\]
		And by definition of $A^{a,b}_a$ and $A^{a,b}_b$ we have
		\[
			x_i|_{\{a,b\}} = x'_i|_{\{a,b\}}
		\]

		For $(x_{-i}, x_i, x'_i) \in \partial_i A^{a,b}_a(p)$, we know that $f((x_{-i}, x_i)) = a$ and $f((x_{-i}, x'_i)) = t$ for $t \in C \backslash \{a\}$. If $t \succ_{x_i} a$ then $x'_i$ is a successful manipulation of $(x_{-i}, x_i)$. Otherwise, $a \succ_{x_i} t$. If this is the case, then we know that $(a, t) \notin Inv_{x_i}$, and because $x_i \le^G_s x'_i$ we have $(a, t) \notin Inv_{x'_i}$, which means $a \succ_{x'_i} t$. Therefore $x_i$ is a successful manipulation of $(x_{-i}, x'_i)$.

		And analogously for $(x_{-i}, x_i, x'_i) \in \partial_i A^{a,b}_b(p)$, either $x'_i$ is a successful manipulation of $(x_{-i}, x_i)$ or $x_i$ is a successful manipulation of $(x_{-i}, x'_i)$.
	\end{proof}

	\begin{lemma}[Lemma 7 of Friedgut]
		\[
			M_i(f) \ge \frac{1}{m!} \left(\frac{m!}{2}\right)^{-n} E_x \left[|\partial_i A^{a,b}_a(p)| + |\partial_i A^{a,b}_b(p)| \right]
		\]
	\end{lemma}

	\begin{proof}
		Recall the definition of $M_i(f)$: given a profile $p \in P$ and vote $p'_i \in V$ chosen uniformly at random, $M_i(f)$ is the probability that $p'_i$ is a successful manipulation of $p$ by voter $i$. Therefore to lower bound $M_i(f)$ we start with $p$ and $p'_i$ chosen uniformly at random. We can think of these as two distinct profiles, $p$ and $p'$, where $p' = (p_{-i}, p'_i)$.

		Clearly $p_{-i}|_{\{a,b\}} = p'_{-i}|_{\{a,b\}}$, but we will have $p_i|_{\{a,b\}} = p'_i|_{\{a,b\}}$ only with probability $\frac{1}{2}$, and we condition the following on this being the case. So we have $p|_{\{a,b\}} = p'|_{\{a,b\}}$.

		By lemma \ref{manipulation-per-edge-in-a}, each $(x_{-i}, x_i, x'_i) \in (\partial_i A^{a,b}_a(p) \cup \partial_i A^{a,b}_b(p))$ corresponds to at least one successful manipulation. Note that if $(x_{-i}, x_i, x'_i) \in \partial_i A^{a,b}_a(p)$ then $(x_{-i}, x'_i, x_i) \notin \partial_i A^{a,b}_a(p)$.

		Therefore we can lower bound $M_i(f)$ by the probability that an edge is in either $\partial_i A^{a,b}_a(p)$ or $\partial_i A^{a,b}_b(p)$. The total possible number of edges is
		\[
			\frac{m!}{2} \cdot \frac{m!}{2} \cdot \left(\frac{m!}{2}\right)^{n-1} = \frac{m!}{2}\left(\frac{m!}{2}\right)^{n}
		\]
		So the probability that a randomly chosen edge is in either $\partial_i A^{a,b}_a(p)$ or $\partial_i A^{a,b}_b(p)$ is
		\[
			\frac{2}{m!} \left(\frac{2}{m!}\right)^{n} \cdot E \left[ |\partial_i A^{a,b}_a(p)| + |\partial_i A^{a,b}_b(p)| \right]
		\]
		Note that we can sum the probabilities for $\partial_i A^{a,b}_a(p)$ and $\partial_i A^{a,b}_b(p)$ because they are disjoint by the definition of the upper edge border; an edge cannot satisfy both $(x_{-i}, x_i) \in A^{a,b}_a(p)$ and $(x_{-i}, x_i) \in A^{a,b}_b(p)$ simultaneously because if $f((x_{-i}, x_i)) = a$ then $f((x_{-i}, x_i)) \ne b$ and vice versa.

		We conditioned our analysis on $p_i = p'_i$, so then, our lower bound becomes
		\[
			M_i(f) \ge \frac{1}{2} \cdot \frac{2}{m!}\left(\frac{2}{m!}\right)^{n} \cdot E \left[ |\partial_i A^{a,b}_a(p)| + |\partial_i A^{a,b}_b(p)| \right]
		\]
		And simplified
		\[
			M_i(f) \ge \frac{1}{m!}\left(\frac{2}{m!}\right)^{n} \cdot E \left[ |\partial_i A^{a,b}_a(p)| + |\partial_i A^{a,b}_b(p)| \right]
		\]
	\end{proof}

	Summing over $i$ we get

	\begin{corollary}[Corollary 1 of Friedgut]
		\[
			\frac{1}{m!} \cdot \left(\frac{m!}{2}\right)^{-n} E_p[|\partial A^{a,b}_a(p)| + |\partial A^{a,b}_b(p)|] \le \sum_i M_i(f)
		\]
	\end{corollary}


\section{Generalized Lemma 8 of Friedgut}

	In this section we will fix candidates $a, b$ and profile $p$, and for the sake of readability we will define the following.

	First, we revise our previous definition to allow us to limit $A$ and $B$ to a specific set of lattice nodes, here referred to as $X$:
	\begin{align*}
		A^{a,b}_c(p, X) = \{x \in X \mid x|_{\{a,b\}} = p|_{\{a,b\}}, f(x) = c\}
	\end{align*}

	Next we define shorthand for $A$ and $B$:
	\begin{align*}
		A(X) &= A^{a,b}_a(p, X) \\
		B(X) &= A^{a,b}_b(p, X) \\
	\end{align*}
	and when we refer to $A$ and $B$ without parameters, we assume they are only limited to valid profiles:
	\begin{align*}
		A &= A(V^n) \\
		B &= B(V^n) \\
	\end{align*}

	We also rename the orderings:
	\begin{align*}
		\le &\textrm{ is } \le_s^G \\
		\le^n &\textrm{ is } (\le_s^G)^n \\
	\end{align*}

	And finally, we recall $(V^n, \le^n)$ is our $n$-dimensional lattice, and that $A$ and $B$ reside in this space:
	\begin{align*}
		A &\subseteq V^n \\
		B &\subseteq V^n \\
	\end{align*}

	\begin{lemma}
		For every disjoint $A, B$ we have that
		\[
			|\partial A| + |\partial B| \ge \left( \frac{2}{m!} \right)^n |A| \cdot |B|
		\]
	\end{lemma}

	\begin{proof}
		We define $A'$ to be a consolidation of $A$ as follows. We start with $A' = A$. We iterate over $i \in \{1, \ldots, n\}$ and $(j_1, \ldots, j_{i-1}, j_{i+1}, \ldots, j_n) \in V^{n-1}$ and $k \in V$. We can view $p = (j_1, \ldots, j_{i-1}, k, j_{i+1}, \ldots, j_n)$ as a profile, i.e. $p \in P$. Let the current dimension be represented by the set $D = \{x | x \in V^n, x_{-i} = p_{-i}\}$. Let $z = \bigjoin D$.

		We know that either $z \notin A(D)$ or $z \notin B(D)$ because $A, B$ are disjoint. If $z \notin A(D)$ then for each $x \in A'(D) \backslash B'(D)$ replace $x$ with any $y \in B' \backslash A'$ (unless $B' \backslash A' = \emptyset$). Otherwise $z \notin B$ then for each $x \in B'(D) \backslash A'(D)$ we replace $x$ with $y \in A' \backslash B'$ (unless $A' \backslash B' = \emptyset$). When we are finished with this, we have that either $A' \subseteq B'$ or $B' \subseteq A'$.

		Note: in the Friedgut paper, he mentions the following condition, which I don't think holds here, but I'm not sure if we need it.
		\begin{align*}
			|\partial A'| &\le |\partial A| \textrm{ and} \\
			|\partial B'| &\le |\partial B| \\
		\end{align*}

		We will now show that
		\begin{align*}
			|A' \backslash A| &\le |\partial A| \textrm{ and} \\
			|B' \backslash B| &\le |\partial B| \\
		\end{align*}
		The only way for an element $y'$ to be in $A' \backslash A$ is that we shifted it during one of the above iterations. We define $y$ to be the original element before it was shifted to $y'$. We also (lazily) define $i$ and $D$ to be the same as they were in whichever iteration caused $y$ to shift to enter $A' \backslash A$. Since $z = \bigjoin D$ and $z \notin A(D)$, we know that $y_i \le z_i$. So the edge $(z_{-i}, z_i, y_i) \in \partial_i A$ and therefore $(z_{-i}, z_i, y_i) \in \partial A$ as well.

		Since every profile in $A' \backslash A$ corresponds to at least one profile in $\partial A$ we know that
		\[
			|A' \backslash A| \le |\partial A|
		\]
		and likewise for $B'$:
		\[
			|B' \backslash B| \le |\partial B|
		\]

		Since for any two votes $v_1, v_2 \in A \cup B$ we have $v_1|_{\{a,b\}} = v_2|_{\{a,b\}}$ we can define a new set
		\[
			P' = \{x \in P \mid x|_{\{a,b\}} = p|_{\{a,b\}}\}
		\]
		and view $A$, $B$, $A'$, and $B'$ as residing in $P'$ without losing any information. This is because any element in $A$, $B$, $A'$, or $B'$ cannot possibly be in $P \backslash P'$: by definition the elements of these sets agree with $p|_{\{a,b\}}$. Clearly $|P'| = (\frac{m!}{2})^n$.

		For any vote $v \in P'$, let $E_{A'}$ be the event that $v$ is in $A'$, and let $E_{B'}$ be the event that $v$ is in $B'$. Then
		\[
			P(E_{A'} \cap E_{B'}) = P(E_{A'}) P(E_{B'}|E_{A'})
		\]
		Clearly
		\begin{align}
			\label{probability-values-1}
			P(E_{A'} \cap E_{B'}) &= \frac{|A' \cap B'|}{|P'|} \\
			\label{probability-values-2}
			P(E_{A'}) &= \frac{|A'|}{|P'|} \\
			\label{probability-values-3}
			P(E_{B'}) &= \frac{|B'|}{|P'|}
		\end{align}
		Since either $A' \subseteq B'$ or $B' \subseteq A'$, we have
		\[
			P(E_{B'}|E_{A'}) \ge P(E_{B'})
		\]
		Therefore
		\[
			P(E_A \cap E_B) \ge P(E_A) P(E_B)
		\]
		So by substitution from equations \ref{probability-values-1}, \ref{probability-values-2}, and \ref{probability-values-3} we get
		\begin{align*}
			\frac{|A' \cap B'|}{(\frac{m!}{2})^n} &\ge \frac{|A'|}{(\frac{m!}{2})^n} \frac{|B'|}{(\frac{m!}{2})^n} \\
			&= \frac{|A|}{(\frac{m!}{2})^n} \frac{|B|}{(\frac{m!}{2})^n}
		\end{align*}
		However $A$ and $B$ are disjoint so
		\[
			A' \cap B' \subseteq (A' \backslash A) \cup (B' \backslash B)
		\]
		which completes the proof as follows
		\begin{align*}
			|A' \cap B'| &\le |A' \backslash A| + |B' \backslash B| \\
			|A' \cap B'| &\le |\partial A| + |\partial B| \\
			\frac{|A||B|}{(\frac{m!}{2})^n} &\le |\partial A| + |\partial B| \\
			\left(\frac{2}{m!}\right)^n |A| \cdot |B| &\le |\partial A| + |\partial B| \\
			|\partial A| + |\partial B| &\ge \left(\frac{2}{m!}\right)^n |A| \cdot |B| \\
		\end{align*}
	\end{proof}

\section{Finished Step 3 of Friedgut}

	Lemma 6, 7, and 8 fit together as follows. First we define the variables $L_6$, $L_7$, and $L_8$ to be variable values that multiply each of the lemmas respectively. The values of these variables will change depending on the value of $m$, so we evaluate the lemmas in terms of these variables to be more general. We can define the lemmas in terms of these variables:
	\begin{align*}
		&M^{a,b} = E[|A||B|] \cdot L_6 & \textrm{lemma 6} \\
		&L_7 \cdot E[|\partial A| + |\partial B|] \le \sum_i M_i & \textrm{lemma 7} \\
		&\frac{1}{L_8} \cdot (|\partial A| + |\partial B|) \ge |A||B| & \textrm{lemma 8}
	\end{align*}

	Now we can solve for the result of step 3.
	\begin{align*}
		M^{a,b} &= E[|A||B|] \cdot L_6 & \textrm{by lemma 6} \\
		M^{a,b} &\le E[|\partial A| + |\partial B|] \cdot \frac{L_6}{L_8} & \textrm{by lemma 8} \\
		M^{a,b} &\le \sum_i M_i \cdot \frac{L_6}{L_7L_8} & \textrm{by lemma 7}
	\end{align*}

	If we can fully generalize this step and capture all of the $v_i$'s our results will, possibly, look like this:
	\begin{align*}
		L_6 &= \left(\frac{m!}{2}\right)^{-2n} \\
		L_7 &= \frac{1}{m!}\left(\frac{m!}{2}\right)^{-n} \\
		L_8 &= \left(\frac{m!}{2}\right)^{-n}
	\end{align*}

	So we have that
	\begin{align*}
		\frac{L_6}{L_7L_8} &= \left(\frac{m!}{2}\right)^{-2n} \cdot m!\left(\frac{m!}{2}\right)^{n} \cdot \left(\frac{m!}{2}\right)^{n} \\
		&= \left(\frac{m!}{2}\right)^{-2n} \cdot m! \cdot \left(\frac{m!}{2}\right)^{2n} \\
		&= m!
	\end{align*}

	And the final result for step 3 becomes
	\begin{align*}
		M^{a,b} &\le \sum_i M_i \cdot m!
	\end{align*}
