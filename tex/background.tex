%!TEX root = thesis.tex

\chapter{Background}

\section{History of Social Choice Theory}

	Election systems are not a recent invention. The earliest democracies resembling what we know today date back to around 508 BC in Athens, Greece. The general idea of elections was used even before that in many other parts of the world \cite{democracybritannica}. In Athens, the assembly was the core of democracy, and any male citizen of at least eighteen years of age was allowed to attend, and therefore, to vote \cite{heinemann1952}. Athenians voted directly on public policy, instead of electing representatives, and voting was done by majority rule. Outside of the assembly, a process known as ostracism was used to exile individuals if necessary. This was done using the plurality voting rule, whereby each man wrote a name on a piece of pottery and the person with the most votes was exiled \cite{oturnbull}.

	Both the majority rule and the plurality systems used in early Greek democracy were very simple. One drawback of these systems is that each voter could only voice a preference for a single candidate. A more accurate way to represent each voter's opinion is with a ranked list of all candidates. This way if candidates tie for first place, the tie can easily be broken by looking at the voter's second choice. If there is still a tie, then the third choice can be taken into account, and so forth. This ranked list is called a preference list.

	In 1770 Jean-Charles de Borda proposed an election system, known now as the Borda count, as a way of electing members of the French Academy of Sciences \cite{borda1781mémoire}. In the Borda count system, each candidate receives points based on their rank in each voter's preference list, i.e. for each first place ranking a candidate will get the most points, for each second place ranking a candidate will get slightly less points, and so on. The winning candidate is the one who receives the greatest total number of points. It was around the time Borda proposed this system that election systems began to be studied academically, though recently it has been discovered that Ramon Llull came up with the Borda count even earlier, in the 13th century \cite{hägele2001llull}.

	Majority rule, plurality, and the Borda count are a few examples of voting rules, but there are many others. Given the large number of voting rules, and that each rule seems to have various strengths and weaknesses, it is useful to compare them to each other. The most obvious criteria for a good election system is fairness \cite{chevaleyre2006issues}. It seems natural that the election system which best represents the constituents' preferences is the best system. Fairness of an election system is easy to recognize if there are only two candidates: the candidate who is preferred by the majority of voters should win. But with a larger number of candidates determining the fairness of an election system is not so obvious.

	Interest in the fairness of voting systems prompted Marquis de Condorcet, a contemporary of Borda, to propose a criterion for voting systems that the winning candidate be the candidate who would win a head-to-head election against each of the other candidates. This is known as the Condorcet criterion and a voting rule satisfying this criterion is completely fair. Unfortunately Condorcet also proved that majority preferences are intransitive in elections with more than two candidates \cite{le1785essai, black1998theory}. In other words, it is possible to have candidate $A \succ $(beats)$ B$, $B \succ C$, $C \succ A$. This means that the Condorcet criterion will not always provide a winner.

	In fact, it has been proven in Arrow's impossibility theorem that no election system can ``fairly'' convert the preferences of voters into a society-wide preference list \cite{arrow1950difficulty}. Arrow gave a list of basic properties that seem obviously required for a fair voting system and proved that no voting system can satisfy all of the properties, hence, no voting system will be completely fair.


\section{History of Manipulation}

	One problem that relates to the issue of fairness is that of manipulation. Manipulation is when an individual purposefully misrepresents his preferences hoping to get a more favorable outcome in the election. For example, if a voter knows that his most preferred candidate has no chance of winning the election, he may instead say that he prefers a different candidate, so that even though his favorite candidate cannot win, at least his second choice candidate has a better chance of winning. This manipulation will benefit the voter but will not benefit society in general, because by lying about his preferences, the voter has skewed the results of the election in his favor. Therefore, researchers attempt to find a way to make manipulation either impossible, or very difficult.

	Unfortunately, it has been shown by the Gibbard-Satterthwaite theorem that all reasonable voting rules are subject to manipulation (or strategic voting or tactical voting). More specifically the Gibbard-Satterthwaite theorem states that every voting rule that is non-dictatorial and non-imposition (see the Preliminaries section for definitions) can be manipulated \cite{gibbard1973manipulation, satterthwaite1975strategy, duggan2000strategic}. This means that we cannot make manipulation impossible, so the best we can do is to make manipulation difficult to perform. This has spawned research that seeks a computational barrier to manipulation, because even if a profitable manipulation exists, it is no use in practice if it is computationally infeasible to find that manipulation.


\section{Recent Results}

	Complexity theorists have analyzed many voting systems using computational complexity as a means of inhibiting manipulation \cite{bartholdi1989computational, hemaspaandra2009hybrid}. Friedgut et al., on the other hand, took a probabilistic approach to this problem \cite{friedgut2008elections}. Instead of studying worst-case manipulation, they performed a probabilistic analysis of random manipulation. That is, instead of a voter intelligently manipulating an election, which can be difficult in terms of worst-case complexity, he simply chooses his manipulation randomly. Friedgut et al. \cite{friedgut2008elections} proved that even an random manipulation will succeed with non-negligible probability. This is significant because no matter how hard it is in the worst-case to find a profitable manipulation, it is trivial to find a random manipulation, which could be enough.

	However, the results of Friedgut et al. \cite{friedgut2008elections} apply only to elections with a maximum of 3 candidates, which is not useful for most practical applications, and is less satisfactory than a general solution from a theoretical standpoint. Building on this work, Xia and Conitzer \cite{xia2008sufficient}, were able to prove the same theorem for any number of candidates, but they assumed different conditions than Friedgut et al. and therefore their work only applies to some of the common voting systems.
