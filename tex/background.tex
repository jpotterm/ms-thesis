%!TEX root = thesis.tex

\chapter{Background}

\section{Brief History of Social Choice Theory}

	Election systems are not a recent invention. The earliest democracies resembling what we know today date back to around 508 BC in Athens, Greece. The general idea of elections was used even before that in many other parts of the world \cite{democracybritannica}. In Athens, the assembly was the core of democracy, and any male citizen of at least eighteen years of age was allowed to attend, and therefore, to vote \cite{heinemann1952}. Athenians voted directly on public policy, instead of electing representatives, and voting was done by majority rule. Outside of the assembly, a process known as ostracism was used to exile individuals if necessary. This was done using the plurality voting rule, whereby each man wrote a name on a piece of pottery and the person with the most votes was exiled \cite{oturnbull}.

	Both the majority rule and the plurality systems used in early Greek democracy were very simple. One drawback of these systems is that each voter could only voice a preference for a single candidate. A more accurate way to represent each voter's opinion is with a ranked list of all candidates. This way if candidates tie for first place, the tie can easily be broken by looking at the voter's second choice. If there is still a tie, then the third choice can be taken into account, and so forth. This ranked list is called a preference list.

	In 1770 Jean-Charles de Borda proposed an election system, known now as the Borda count, as a way of electing members of the French Academy of Sciences \cite{borda1781mémoire}. In the Borda count system, each candidate receives points based on their rank in each voter's preference list, i.e. for each first place ranking a candidate will get the most points, for each second place ranking a candidate will get slightly less points, and so on. The winning candidate is the one who receives the greatest total number of points. It was around the time Borda proposed this system that election systems began to be studied academically, though recently it has been discovered that Ramon Llull came up with the Borda count even earlier, in the 13th century \cite{hägele2001llull}.

	Majority rule, plurality, and the Borda count are a few examples of voting rules, but there are many others. Given the large number of voting rules, and that each rule seems to have various strengths and weaknesses, it is useful to compare them to each other. The most obvious criteria for a good election system is fairness \cite{chevaleyre2006issues}. It seems natural that the election system which best represents the constituents' preferences is the best system. Fairness of an election system is easy to recognize if there are only two candidates: the candidate who is preferred by the majority of voters should win. But with a larger number of candidates determining the fairness of an election system is not so obvious.

	Interest in the fairness of voting systems prompted Marquis de Condorcet, a contemporary of Borda, to propose a criterion for voting systems that the winning candidate be the candidate who would win a head-to-head election against each of the other candidates (1785). This is known as the Condorcet criterion and a voting rule satisfying this criterion is completely fair. Unfortunately Condorcet also proved that majority preferences are intransitive in elections with more than two candidates \cite{le1785essai, black1998theory}. In other words, it is possible to have candidate $A \succ $(beats)$ B$, $B \succ C$, $C \succ A$. This means that the Condorcet criterion will not always provide a winner.

	In 1876, Charles Dodgson proposed an election system similar to the Condorcet criterion, except that it solves the problem of intransitive majority preferences by declaring the winner to be whichever candidate satisfies the Condorcet criterion with the fewest changes in voters’ preferences \cite{dodgson1876method}.

	The work done by Condorcet along with later work by Arrow is widely regarded as being foundational to the modern field of Social Choice Theory, and marks a transition from viewing social choice as a purely practical problem to a more rigorous theoretical study.


\section{History of Manipulation}

	One problem that relates to the issue of fairness is that of manipulation. Manipulation is when an individual purposefully misrepresents his preferences hoping to get a more favorable outcome in the election. For example, if a voter knows that his most preferred candidate has no chance of winning the election, he may instead say that he prefers a different candidate, so that even though his favorite candidate cannot win, at least his second choice candidate has a better chance of winning. This manipulation will benefit the voter but will not benefit the society in general, because by lying about his preferences, the voter has skewed the results of the election in his favor. Therefore, it is beneficial to search for ways to avoid manipulation in election systems.

	One way to avoid manipulation could be to devise a voting rule that is non-manipulable. Unfortunately, in 1950, Arrow's impossibility theorem \cite{arrow1950difficulty} proved that no election system can ``fairly'' convert the preferences of voters into a society-wide preference list. Arrow gave a list of basic properties that seem obviously required for a fair voting system and proved that no voting system can satisfy all of the properties, hence, no voting system will be completely fair.

	In 1973 the Gibbard-Satterthwaite theorem gave a similar result for election systems that choose a single winner: that all reasonable voting rules are subject to manipulation (or strategic voting or tactical voting). More specifically the Gibbard-Satterthwaite theorem states that every voting rule that is non-dictatorial and non-imposition (see the Preliminaries section for definitions) can be manipulated \cite{gibbard1973manipulation, satterthwaite1975strategy, duggan2000strategic}.

	This means that we cannot make manipulation impossible by cleverly devising a voting rule, which is a rather disappointing prospect. More recently, however, the field of Computational Social Choice has, instead of trying to make manipulation impossible, endeavored to make it computationally intractable \cite{chevaleyre2007short}. This has spawned research that seeks a computational barrier to manipulation, because even if a profitable manipulation exists, it is no use in practice if it is computationally infeasible to find that manipulation.
