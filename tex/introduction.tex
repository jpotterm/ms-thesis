%!TEX root = thesis.tex

\chapter{Introduction}

We study manipulation in election systems, looking specifically at random manipulation, and basing our work on the results of Friedgut, Kalai, and Nisan \cite{friedgut2008elections} and Xia and Conitzer \cite{xia2008sufficient}. Note that we cite the original work by Friedgut et al. \cite{friedgut2008elections} to compare it to the work by Xia and Conitzer \cite{xia2008sufficient}. However, a journal version has recently been published which loosens some of the original constraints \cite{friedgut2011quantitative} which we will describe in the Topic Area section. We will analyze these results and attempt to improve upon them by generalizing the results of Friedgut, Kalai, and Nisan, or loosening the conditions of Xia and Conitzer.  These authors condition their results on voting rules having certain properties, and we intend to look at what happens when these conditions are replaced by stronger or weaker conditions. We would also like to tighten the bounds given by these two papers by a constant factor or by making them depend of the number of candidates.

Though election systems have traditionally been used in politics, they have also been applied to many other situations. Election systems are used in schools to elect board members, and they are used in businesses for stock holders to vote on the direction of a company. Recently, artificial intelligence systems have been holding elections when a group of intelligent agents need to reach a common consensus \cite{ephrati1991clarke, ephrati1993multi, pennock2000social, dwork2001rank, fagin2003efficient}.

Because election systems are so widespread and are crucial for the functioning of society, it is beneficial to study their nature along with their strengths and weaknesses. Election systems have been studied academically since as early as the 13th century \cite{hägele2001llull}, but more recently they have been scrutinized by the computer science community in the fields of theory and artificial intelligence.


\section{Overview of Hypotheses}

We will attempt to generalize the work of Friedgut et al. \cite{friedgut2008elections} and Xia and Conitzer \cite{xia2008sufficient}. This will include attempting to extend the results of Friedgut et al. \cite{friedgut2008elections} to include elections with more than three candidates. We will also extend the results of Xia and Conitzer \cite{xia2008sufficient} to capture more common voting systems by relaxing their conditions and still proving the same results. We believe that the worst case bound for random manipulation can be tightened by a constant or by depending on the number of candidates.
