%!TEX root = thesis.tex

\chapter{Conclusion}

	We have attempted to prove the following main result in order to extend the results of Friedgut, Kalai, and Nisan in a straightforward manner, and along the same lines of reasoning as the original proof. Our main theorem is

	\begin{theorem}[Main Result]
		There exists a constant $C > 0$ such that for every $\epsilon > 0$ the following holds. If $f$ is a neutral SCF for $n$ voters over 3 alternatives and $\Delta(f, g) > \epsilon$ for any dictatorship $g$, then $f$ has total manipulatiblity: $\sum^n_{i=1} M_i(f) \ge \frac{(C\epsilon)^{\lfloor m/3 \rfloor}}{m!}$.
	\end{theorem}

	Unfortunately, due to time constraints we were unable to prove (or disprove) Lemma \ref{friedgut-lemma-8} because of the problem noted in the attempted proof for that lemma. We conjecture that this can be proven without many changes to our attempted proof, and we leave this as our first open problem.

	Since our main result relies on Lemma \ref{friedgut-lemma-8}, we have only been able to show the following two lemmas.

	\begin{lemma}
		Let $C$ be a set of alternatives. Let $a, b \in C$ be any two alternatives. Let $m = |C|$ and let $n$ be the number of voters. Let $f$ be a SCF. We have
		\[
			M^{a,b}(f) = E_{p \in L(C)^n} \left[ \frac{|A^{a,b}_a(p)|}{\left(\frac{m!}{2}\right)^n} \cdot \frac{|A^{a,b}_b(p)|}{\left(\frac{m!}{2}\right)^n} \right]
		\]
	\end{lemma}

	\begin{lemma}
		\[
			M_i(f) \ge \frac{1}{m!} \left(\frac{m!}{2}\right)^{-n} E_x \left[|\partial_i A^{a,b}_a(p)| + |\partial_i A^{a,b}_b(p)| \right]
		\]
	\end{lemma}

	We made much progress in proving Lemma \ref{friedgut-lemma-8}, but were unable to bring it to completion. If it can be proven, then our main result will follow.

\section{Open Problems}
	The following open problems are potential areas of further research that follow from this thesis.

	\begin{enumerate}
		\item Complete the proof of Lemma \ref{friedgut-lemma-8} by resolving Open Problem \ref{lemma-8-open-problem}, which is stated in that lemma.
		\item Throughout this thesis we have been concerned only with average-case complexity based on a normal distribution of votes. This is common practice and it is a useful distribution, but in real-world elections the distribution is very rarely a normal distribution, so we leave the task of investigating the impact of a more realistic distribution for further research.
		\item Our bounds are certainly not tight, and neither are the bounds of Mossel and R\'{a}cz \cite{mossel2011quantitative}. It would be ideal to find tight bounds.
		\item This work deals with a lower bound on average case manipulability, but some subsets of SCFs may be close to this lower bound, while others may be significantly higher. It would be useful to investigate the average case manipulability of certain subsets of SCFs to see how close they are to this generic lower bound. This is one of the open problems stated by Mossel and R\'{a}cz \cite{mossel2011quantitative}.
	\end{enumerate}
