%!TEX root = thesis.tex

\chapter{Preliminaries}

	In this chapter we present up the formal definitions we will need for the rest of the thesis. These serve as a reference and an introduction to the technical work we have done in the following chapters. Some definitions are very basic indeed and many will already be known to the reader, and so they are stated explicitly here simply to provide maximum clarity and a solid foundation upon which to present the rest of our work.

	For additional definitions which are out of the scope of this thesis the reader may refer to textbooks on set theory \cite{kunen1980set} and lattice theory \cite{birkhoff1995lattice}.

\section{Definitions}

	We will begin with basic definitions regarding set theory and lattice theory, and then towards the end of the section we will transition to definitions from social choice theory.

	\begin{definition}
		A \emph{permutation} of a set $X$ is a bijective function from $X$ to $X$.
	\end{definition}

	\begin{definition}
		A \emph{total ordering} over a set $X$ is a binary relation on $X$ which is antisymmetric, transitive, and total.
	\end{definition}

	Although technically permutations and total orderings are different constructs (bijective functions versus binary relations), they often have similar applications. For example, given a totally-ordered set $X$ and a permutation $\sigma$ on $X$, we can construct a total ordering $<_R = \{(x,y) \mid x, y \in X, \sigma^{-1}(x) < \sigma^{-1}(y)\}$ which is similar to $\sigma$. Likewise, for any well-ordered set $X$ and a total ordering, $<_R$, over $X$, one could construct a permutation $\sigma$ from $<_R$ where $\sigma : X \to X$ such that $\sigma(x) = y$ iff
	\begin{align*}
		|\{z \in X \mid z < x\}| = |\{z \in X \mid z <_R y\}|.
	\end{align*}

	For countable sets we will sometimes view permutations and total orders as sequences of elements, using a subscript notation, provided our meaning is clear from context.

	\begin{definition}
		We use $S(X)$ to denote the set of all permutations of $X$.
	\end{definition}

	\begin{definition}
		We use $L(X)$ to denote the set of all total orders over $X$.
	\end{definition}

	\begin{definition}
		We define a \emph{poset}, or \emph{partially ordered set}, to be $(X, \le)$ where $X$ is a set, and $\le$ is a binary relation on $X$ which is antisymmetric, transitive, and reflexive. $\le$ is called a partial ordering because of the fact that not every pair of elements in $X$ needs to be related by $\le$, as opposed to a total ordering which must relate every pair.
	\end{definition}

	\begin{definition}
		For any poset $(P, \le)$, a \emph{lower bound} of a subset $X \subseteq P$ is an element $a \in P$ such that $a \le x$ for every $x \in X$. A \emph{greatest lower bound} is a \emph{lower bound} that is greater than or equal to every other \emph{lower bound}. We denote this \emph{greatest lower bound} as $\inf_P X$ calling it the \emph{infimum} \cite{birkhoff1967lattice} and also as $\bigmeet_P X$ calling it the \emph{meet}. When $X$ contains only two elements, we can use the meet as a binary operator: $\bigmeet_P \{a, b\} = a \meet_P b$. When $P$ is obvious from context we will simply write $\inf X$ or $\bigmeet X$. If the infimum exists, it is unique because posets are antisymmetric. The infimum is the same as the supremum in the inverse order.
	\end{definition}

	\begin{definition}
		For any poset $(P, \le)$, an \emph{upper bound} of a subset $X \subseteq P$ is an element $a \in P$ such that $a \ge x$ for every $x \in X$. A \emph{least upper bound} is an \emph{upper bound} that is less than or equal to every other \emph{upper bound}. We denote this \emph{least upper bound} as $\sup_P X$ calling it the \emph{supremum} \cite{birkhoff1967lattice} and also as $\bigjoin_P X$ calling it the \emph{join}. When $X$ contains only two elements, we can use the join as a binary operator: $\bigjoin_P \{a, b\} = a \join_P b$. When $P$ is obvious from context we will simply write $\sup X$ or $\bigjoin X$. If the supremum exists, it is unique because posets are antisymmetric. The supremum is the same as the infimum in the inverse order.
	\end{definition}

	\begin{definition}
		\label{lattice-definition}
		A poset, $(P, \le)$, is a \emph{lattice} if for any $x, y \in P$ both the meet and join of $x$ and $y$ exist. Note that the meet and join are unique by definition (if they exist).
	\end{definition}

	\begin{definition}
		\label{transitive-closure-definition}
		The \emph{transitive closure} of a binary relation $R$ on a set $X$ is the transitive relation $R^t$ on $X$ such that $R \subseteq R^t$ and $R^t$ is minimal \cite[p. 337]{lidl1998applied}.
	\end{definition}

	\begin{definition}
		\label{inversion-definition}
		For any poset $(P, \le)$, let $\sigma$ be a permutation of $P$. We define the \emph{inversions} of $\sigma$ to be a binary relation $\Inv_{\sigma}$ on $P$:
		\[
			\Inv_{\sigma} = \{(i,j) \mid i, j \in P, i < j, \sigma^{-1}(i) > \sigma^{-1}(j)\}.
		\]
		We can read $i \Inv_{\sigma} j$ as ``$i$ is inverted with $j$ in $\sigma$''. $\Inv$ is a transitive relation because for any $i,j,k \in P$ if $i \Inv_{\sigma} j$ and $j \Inv_{\sigma} k$ then $i < j < k$ and $\sigma^{-1}(i) > \sigma^{-1}(k) > \sigma^{-1}(k)$ which means that $i \Inv_{\sigma} k$.

		In addition, let $(X, \le')$ be a lattice such that the elements of $X$ are permutations of $P$. For any $\sigma, \pi \in X$ we have \cite{markowsky1994permutation}:
		\[
			\Inv_{\sigma \meet \pi} = (\Inv_{\sigma} \cup \Inv_{\pi})^t.
		\]
	\end{definition}

	\begin{definition}
		\label{kendall-tau-definition}
		For any poset $(P, \le)$, let $\sigma$ and $\pi$ be a permutations of $P$. We define the \emph{Kendall tau distance}, $K$, between $\sigma$ and $\pi$ to be the number of adjacent swaps one would have to make to get from $\sigma$ to $\pi$ or vice versa. More formally:
		\[
			K(\sigma, \pi) = \sum_{\{i, j\} \in P} \overline{K}_{i, j}(\sigma, \pi)
		\]
		where
		\[
			\overline{K}_{i,j}(\sigma, \pi) =
				\begin{cases}
					0 & \textrm{ if } i \textrm{ and } j \textrm{ are in the same order in } \sigma \textrm{ and } \pi \\
					1 & \textrm{ if } i \textrm{ and } j \textrm{ are in the opposite order in } \sigma \textrm{ and } \pi
				\end{cases}
		\]
		Alternatively we can define the Kendall tau distance in terms of inversions
		\[
			K(\sigma, \pi) = |\Inv_{\sigma} \symmetricdifference \Inv_{\pi}|
		\]
		with $\symmetricdifference$ denoting symmetric difference.
	\end{definition}

	\begin{definition}
		For any poset $(P, \le)$, let $x,y \in P$. We say that $x$ is a \emph{predecessor} of $y$ if $x < y$. We say that $x$ is a \emph{direct predecessor} of $y$ if $x$ is the greatest predecessor of $y$.
	\end{definition}

	\begin{definition}
		For any poset $(P, \le)$, let $x,y \in P$. We say that $x$ is a \emph{successor} of $y$ if $x > y$. We say that $x$ is a \emph{direct successor} of $y$ if $x$ is the least successor of $y$.
	\end{definition}

	In the next couple of definitions and many of the lemmas in this section, we will be investigating lattices whose elements are permutations of a set. That is, given a set $Y$, we will study some of the properties of the lattice $(S(Y), \le)$.

	\begin{definition}
		For any partial order $\le_x$ over a set $X$, and $a, b \in X$, we define the following additional partial orders over $X$:
		\begin{align*}
			a \ge_x b &\equiv b \le_x a \\
			a =_x b &\equiv a \le_x b \text{ and } b \le_x a \\
			a \ne_x b &\equiv \text{not } a =_x b \\
			a <_x b &\equiv a \le_x b \text{ and } a \ne_x b \\
			a >_x b &\equiv a \ge_x b \text{ and } a \ne_x b.
		\end{align*}
		We will treat these additional partial orders as implicitly defined, and will not define them explicitly for each partial order individually.
	\end{definition}

	\begin{definition}[$\le_s$]
		\label{partial-order-s-definition}
		Let $(P, \le)$ be a poset and let $X = S(P)$. We define the partial ordering $\le_s$ on $X$ such that for all $\sigma, \pi \in X$:
		\[
			\sigma \le_s \pi \iff \Inv_{\sigma} \subseteq \Inv_{\pi}.
		\]
	\end{definition}

	\begin{definition}[$X^{ij}, \le^{ij}$]
		\label{identified-lattice-definition}
		Let $Y$ be a set and let $X = S(Y)$. Let $(X, \le)$ be a lattice. For any $i,j \in Y$ we define
		\[
			X^{ij} = \{ x \in X \mid x^{-1}(i) < x^{-1}(j) \}.
		\]
		We then define the partial ordering, $\le^{ij}$, over $X^{ij}$ such that for $x, y \in X^{ij}$:
		\[
			x \le^{ij} y \iff x \le y
		\]
	\end{definition}

	We will now introduce some definitions having to do with social choice theory. Throughout this paper we will use $n$ to represent the number of voters in an election, and $m$ to represent the number of alternatives (candidates).

	\begin{definition}
		Let $C = \{1, \ldots, m\}$ be the set of all \emph{alternatives} (candidates). We define the set of all \emph{preference profiles} to be $P = L(C)^n$.
	\end{definition}

	\begin{definition}
		We define the set of all \emph{preference lists} to be $V = L(C)$. We can also view a preference list as a permutation on $C$; it will be obvious from context which approach we are using.
	\end{definition}

	\begin{definition}
		We define a \emph{voting rule}, or \emph{social choice function} (SCF), to be a function $f : P \to C$.
	\end{definition}

	\begin{definition}
		We define an \emph{election} to be simply a voting rule paired with a profile: $(f, p)$ where $f$ is a voting rule and $p \in P$.
	\end{definition}

	\begin{definition}
		Let $v \in V$ be a preference list, and let $x, y \in C$ be two alternatives. Since $v$ is actually a total ordering, we denote $(x, y) \in v$ by
		\[
			x <_v y
		\]
		and if this is the case we view $x$ as being ranked above $y$ in $v$ and we say that $x$ beats $y$, and denote this as
		\[
			x \succ_v y.
		\]
		We view $x$ as being ranked below $y$ in $v$ if
		\[
			x >_v y
		\]
		and we would say that $x$ is beaten by $y$, we denote this as
		\[
			x \prec_v y
		\]
	\end{definition}

	\begin{definition}
		\label{preference-restriction-definition}
		For a set of candidates $D \subseteq C$, for a preference list $v \in V$ and a preference profile $p \in P$ we denote $v$ and $p$ \emph{restricted to} $D$ by $v|_D$ and $p|_D$ respectively. $v|_D$ means $v$ after all the candidates who are not in $D$ have been removed from the preference list. $p|_D$ means that every preference list in $p$ has been restricted to $D$.

		We will sometimes wish to restrict to every alternative except those in $D$ in which case we will write $v|_{\overline{D}}$ where the universe is understood to be $C$.

		Since we will often use restriction when comparing two preference lists we will write $x|_D = y|_D$ simply as $x =_D y$.
	\end{definition}

	\begin{definition}
		For any sequence $v$, and $i \in \{1, \ldots, |v|\}$ we will denote by $v_{-i}$, $v$ with $v_i$ removed.
	\end{definition}

	\begin{definition}
		\label{manipulation-definition}
		A \emph{successful manipulation} (or \emph{profitable manipulation}) by voter $i$ of a SCF $f$ at profile $x$ is a preference list $x'_i$ such that
		\[
			f((x_{-i}, x'_i)) \succ_i f((x_{-i}, x_i)).
		\]
	\end{definition}
