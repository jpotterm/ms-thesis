%!TEX root = thesis.tex

\chapter{Preliminaries}

	In this chapter we present the formal definitions we will need for the rest of this thesis. These serve as a reference as well as an introduction to the technical work contained in the following chapters. Some definitions are very basic and many will already be known to the reader; however, they are stated explicitly here to provide maximum clarity and a solid foundation upon which to present the rest of our work.

	For additional definitions, which are out of the scope of this thesis, the reader may refer to textbooks on set theory \cite{kunen1980set} and lattice theory \cite{birkhoff1995lattice}.

\section{Definitions}

	We will begin with basic definitions regarding set theory and lattice theory, and then towards the end of this section we will transition to definitions from social choice theory.

	\begin{definition}
		A \emph{permutation} of a set $X$ is a bijective function from $X$ to $X$.
	\end{definition}

	\begin{definition}
		A \emph{total ordering} over a set $X$ is a binary relation on $X$ which is antisymmetric, transitive, and total.
	\end{definition}

	Although technically speaking permutations and total orderings are different constructs (bijective functions versus binary relations), they often have similar applications. For example, given a totally-ordered set $X$ and a permutation $\sigma$ on $X$, we can construct a total ordering $<_R = \{(x,y) \mid x, y \in X, \sigma^{-1}(x) < \sigma^{-1}(y)\}$ which is structurally analogous to $\sigma$. Likewise, for any well-ordered set $X$ and a total ordering $<_R$ over $X$, one could construct a permutation $\sigma$ from $<_R$ where $\sigma : X \to X$ such that $\sigma(x) = y$ iff
	\begin{align*}
		|\{z \in X \mid z < x\}| = |\{z \in X \mid z <_R y\}|.
	\end{align*}

	For countable sets we will sometimes view permutations and total orders as sequences, using a subscript notation, provided our meaning is clear from context.

	\begin{definition}
		We use $S(X)$ to denote the set of all permutations of $X$.
	\end{definition}

	\begin{definition}
		We use $L(X)$ to denote the set of all total orders over $X$.
	\end{definition}

	\begin{definition}
		We define a \emph{poset}, or \emph{partially ordered set}, to be $(X, \le)$, where $X$ is a set and $\le$ is a binary relation on $X$ that is antisymmetric, transitive, and reflexive. $\le$ is called ``partial'' because not every pair of elements in $X$ needs to be related by $\le$, as opposed to a total ordering which must relate every pair.
	\end{definition}

	\begin{definition}
		For any partial order $\le_x$ over a set $X$ and $a, b \in X$, we define the following additional relations over $X$:
		\begin{align*}
			a \ge_x b &\equiv b \le_x a, \\
			a =_x b &\equiv a \le_x b \text{ and } b \le_x a, \\
			a \ne_x b &\equiv \text{not } a =_x b, \\
			a <_x b &\equiv a \le_x b \text{ and } a \ne_x b, \\
			a >_x b &\equiv a \ge_x b \text{ and } a \ne_x b.
		\end{align*}
		We will treat these additional relations as implicitly defined, and will not define them explicitly for each partial order individually.
	\end{definition}

	\begin{definition}
		For any poset $(P, \le)$, a \emph{lower bound} of a subset $X \subseteq P$ is an element $a \in P$ such that $a \le x$ for every $x \in X$. A \emph{greatest lower bound} is a \emph{lower bound} that is greater than or equal to every other \emph{lower bound}. We denote this \emph{greatest lower bound} as $\inf_P X$, calling it the \emph{infimum} \cite{birkhoff1967lattice} and also as $\bigmeet_P X$, calling it the \emph{meet}. When $X$ contains only two elements, we can use the meet as a binary operator: $\bigmeet_P \{a, b\} = a \meet_P b$. When $P$ is obvious from context we will simply write $\inf X$ or $\bigmeet X$. If the infimum exists, it is unique because posets are antisymmetric.
	\end{definition}

	\begin{definition}
		For any poset $(P, \le)$, an \emph{upper bound} of a subset $X \subseteq P$ is an element $a \in P$ such that $a \ge x$ for every $x \in X$. A \emph{least upper bound} is an \emph{upper bound} that is less than or equal to every other \emph{upper bound}. We denote this \emph{least upper bound} as $\sup_P X$, calling it the \emph{supremum} \cite{birkhoff1967lattice} and also as $\bigjoin_P X$, calling it the \emph{join}. When $X$ contains only two elements, we can use the join as a binary operator: $\bigjoin_P \{a, b\} = a \join_P b$. When $P$ is obvious from context we will simply write $\sup X$ or $\bigjoin X$. If the supremum exists, it is unique because posets are antisymmetric. The supremum is the same as the infimum in the inverse order, and vice versa.
	\end{definition}

	\begin{definition}
		\label{lattice-definition}
		A poset, $(P, \le)$, is a \emph{lattice} if for any $x, y \in P$ both the meet and join of $\{x, y\}$ exist. Note that the meet and join are unique by definition (if they exist).
	\end{definition}

	\begin{definition}
		\label{transitive-closure-definition}
		The \emph{transitive closure} of a binary relation $R$ on a set $X$ is the transitive relation $R^t$ on $X$ such that $R \subseteq R^t$ and $R^t$ is minimal \cite[p. 337]{lidl1998applied}.
	\end{definition}

	\begin{definition}
		\label{inversion-definition}
		For any poset $(P, \le)$, let $\sigma$ be a permutation of $P$. We define the \emph{inversions} of $\sigma$ to be a binary relation $\Inv_{\sigma}$ on $P$:
		\[
			\Inv_{\sigma} = \{(i,j) \mid i, j \in P, i < j, \sigma^{-1}(i) > \sigma^{-1}(j)\}.
		\]
		We can read $i \Inv_{\sigma} j$ as ``$i$ is inverted with $j$ in $\sigma$''.

		$\Inv$ is a transitive relation because for $n \ge 1$ and $i, k, j \in P$ we have that if
		\[
			(i, k), (k, j) \in \Inv_{\sigma}
		\]
		then both of the following hold:
		\begin{align*}
			i < &k < j \\
			\sigma^{-1}(i) > \sigma^{-1}&(k) > \sigma^{-1}(j).
		\end{align*}
		And therefore $(i, j) \in \Inv_{\sigma}$.
	\end{definition}

	\begin{definition}
		\label{kendall-tau-definition}
		For any poset $(P, \le)$, let $\sigma$ and $\pi$ be permutations of $P$. We define the \emph{Kendall tau distance}, $K$, between $\sigma$ and $\pi$ to be the number of adjacent swaps necessary to get from $\sigma$ to $\pi$ or vice versa. More formally:
		\[
			K(\sigma, \pi) = \sum_{\{i, j\} \in P} \overline{K}_{i, j}(\sigma, \pi)
		\]
		where
		\[
			\overline{K}_{i,j}(\sigma, \pi) =
				\begin{cases}
					0 & \textrm{ if } i \textrm{ and } j \textrm{ are in the same order in } \sigma \textrm{ and } \pi \\
					1 & \textrm{ if } i \textrm{ and } j \textrm{ are in the opposite order in } \sigma \textrm{ and } \pi
				\end{cases}
		\]
		Alternatively we can define the Kendall tau distance in terms of inversions:
		\[
			K(\sigma, \pi) = |\Inv_{\sigma} \symmetricdifference \Inv_{\pi}|,
		\]
		with $\symmetricdifference$ denoting symmetric difference.
	\end{definition}

	\begin{definition}
		For any poset $(P, \le)$, let $x,y \in P$. We say that $x$ is a \emph{predecessor} of $y$ if $x < y$. We say that $x$ is a \emph{direct predecessor} of $y$ if $x$ is the greatest predecessor of $y$.
	\end{definition}

	\begin{definition}
		For any poset $(P, \le)$, let $x,y \in P$. We say that $x$ is a \emph{successor} of $y$ if $x > y$. We say that $x$ is a \emph{direct successor} of $y$ if $x$ is the least successor of $y$.
	\end{definition}

	In the next couple of definitions and many of the lemmas in this section, we will be investigating lattices whose elements are permutations of a set. That is, given a set $Y$, we will study some of the properties of the lattice $(S(Y), \le)$.

	\begin{definition}[$\le_s$]
		\label{partial-order-s-definition}
		Let $(Y, \le)$ be a poset and let $X = S(Y)$. We define the partial ordering $\le_s$ on $X$ such that for all $\sigma, \pi \in X$:
		\[
			\sigma \le_s \pi \iff \Inv_{\sigma} \supseteq \Inv_{\pi}.
		\]

		In addition, if $(X, \le_s)$ is a lattice, then for any $\sigma, \pi \in X$ we also have \cite{markowsky1994permutation}:
		\[
			\Inv_{\sigma \meet \pi} = (\Inv_{\sigma} \cup \Inv_{\pi})^t.
		\]
		Note that $^t$ indicates the transitive closure as defined in Definition \ref{transitive-closure-definition}.
	\end{definition}

	\begin{definition}[$X^{ij}, \le^{ij}$]
		\label{identified-lattice-definition}
		Let $Y$ be a set, let $X = S(Y)$, and let $(X, \le)$ be a lattice. For any $i,j \in Y$ we define
		\[
			X^{ij} = \{ x \in X \mid x^{-1}(i) < x^{-1}(j) \}.
		\]
		We then define the partial ordering, $\le^{ij}$, over $X^{ij}$ such that for all $x, y \in X^{ij}$:
		\[
			x \le^{ij} y \iff x \le y
		\]
	\end{definition}

	We will now introduce some definitions having to do with social choice theory. Throughout this paper we will use $n$ to represent the number of voters in an election, and $m$ to represent the number of alternatives (candidates).

	\begin{definition}
		Let $C = \{1, \ldots, m\}$ be the set of all \emph{alternatives} (candidates).
	\end{definition}

	\begin{definition}
		We define a \emph{preference list} to be a total order over $C$. Therefore the set of all preference lists is $L(C)$. Alternatively we can view a preference list as a permutation on $C$, and it will be obvious from context which approach we are using.
	\end{definition}

	\begin{definition}
		We define a \emph{preference profile} to be $p \in L(C)^n$. And we define $P = L(C)^n$ to be the set of all preference profiles.
	\end{definition}

	\begin{definition}
		We define a \emph{voting rule}, or \emph{social choice function} (SCF), to be a function $f : P \to C$.
	\end{definition}

	\begin{definition}
		We define an \emph{election} to be simply a voting rule paired with a profile: $(f, p)$ where $f$ is a voting rule and $p \in P$.
	\end{definition}

	\begin{definition}
		Let $v \in L(C)$ be a preference list, and let $x, y \in C$ be two alternatives. Since $v$ is actually a total ordering, we denote $(x, y) \in v$ by
		\[
			x <_v y
		\]
		and if this is the case we view $x$ as being ranked above $y$ in $v$ and we say that $x$ beats $y$, and denote this as
		\[
			x \succ_v y.
		\]
		We view $x$ as being ranked below $y$ in $v$ if
		\[
			x >_v y
		\]
		and we would say that $x$ is beaten by $y$, we denote this as
		\[
			x \prec_v y
		\]
	\end{definition}

	\begin{definition}
		\label{preference-restriction-definition}
		For a set of candidates $D \subseteq C$, for a preference list $v \in L(C)$ and a preference profile $p \in P$ we denote $v$ and $p$ \emph{restricted to} $D$ by $v|_D$ and $p|_D$ respectively. $v|_D$ means $v$ after all the candidates who are not in $D$ have been removed. $p|_D$ means that every preference list in $p$ has been restricted to $D$.

		We will sometimes wish to restrict to every alternative except those in $D$ in which case we will write $v|_{\overline{D}}$ where the universe is understood to be $C$.

		Since we will often use restriction when comparing two preference lists we will write $x|_D = y|_D$ simply as $x =_D y$.
	\end{definition}

	\begin{definition}
		For any sequence $x$, and $i \in \{1, \ldots, |x|\}$ we will denote by $x_{-i}$, $x$ with $x_i$ removed. Therefore $(x_{-i}, x'_i)$ will mean a sequence equal to $x$ except with $x'_i$ replacing $x_i$.
	\end{definition}

	\begin{definition}
		\label{manipulation-definition}
		A \emph{successful manipulation} (or \emph{profitable manipulation}) by voter $i$ of a SCF $f$ at profile $p$ is a preference list $p'_i$ such that
		\[
			f((p_{-i}, p'_i)) \succ_i f((p_{-i}, p_i)).
		\]
	\end{definition}

	\begin{definition}
		\label{scf-distance-definition}
		The distance between two SCFs is defined to be the fraction of inputs on which they differ:
		\[
			\mathbf{D}(f, g) = \mathbb{P}(f (\sigma) \ne g(\sigma)).
		\]
		When dealing with a set of SCFs, $G$, we take the minimum distance over all SCFs in that set:
		\[
			\mathbf{D}(f, G) = \min_{g \in G} \mathbf{D}(f, g).
		\]
		For example, we will sometimes say that a SCF, $f$, is $\epsilon$-far from dictatorship. This means that $\mathbf{D}(f, DICT) > \epsilon$, where $DICT$ is the set of SCFs that are dictatorships.
	\end{definition}

	\begin{definition}
		A SCF is \emph{neutral} if its result does not change with the permutation of the alternatives. Formally, let $f$ be a SCF, let $\sigma$ be a permutation over the set of alternatives, and let $\pi$ be a function which applies $\sigma$ to each alternative in a preference list:
		\[
			\pi(v) = (\sigma(v_1), \ldots, \sigma(v_m)).
		\]
		Then $f$ is neutral iff
		\[
			f(\pi(p_1), \ldots, \pi(p_n)) = \sigma(f(p_1, \ldots, p_n)).
		\]
	\end{definition}
