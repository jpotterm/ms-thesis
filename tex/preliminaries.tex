%!TEX root = thesis.tex

\chapter{Preliminaries}

	\begin{definition}
		A \emph{permutation} of a set $X$ is a bijective function from $X$ to $X$.
	\end{definition}

	\begin{definition}
		We use $L(X)$ to denote the set of all total orders over $X$.
	\end{definition}

	For countable sets we will sometimes view permutations and total orders as sequences of elements using a subscript notation as long as our meaning is clear from the context.

	\begin{definition}
		Throughout this paper we will use $n$ to represent the number of voters in an election, and $m$ to represent the number of candidates. Let $C = \{1, \ldots, m\}$ be the set of all \emph{alternatives} (candidates). We define the set of all \emph{preference lists} to be $V = L(C)$. We can also view a preference list as a permutation on $C$; it will be obvious from context which approach we are using. We define the set of all \emph{preference profiles} to be $P = L(C)^n$. We define a \emph{voting rule}, or \emph{social choice function} (SCF), to be a function $f : P \to C$. And lastly we define an \emph{election} to be simply a voting rule paired with a profile: $(f, p)$ where $f$ is a voting rule and $p \in P$.
	\end{definition}

	\begin{definition}
		A \emph{successful manipulation} (or \emph{profitable manipulation}) by voter $i$ of a SCF $f$ at profile $x$ is a preference $x'_i$ such that
		\[
			f((x_{-i}, x'_i)) \succ_i f((x_{-i}, x_i))
		\]
	\end{definition}

	\begin{definition}
		Let $v \in V$ be a preference list, and let $x, y \in C$ be two alternatives. Since $v$ is actually a total ordering, we denote $(x, y) \in v$ by
		\[
			x <_v y
		\]
		and if this is the case we view $x$ as being ranked above $y$ in $v$ and we say that $x$ beats $y$, and denote this as
		\[
			x \succ_v y
		\]
		We view $x$ as being ranked below $y$ in $v$ if
		\[
			x >_v y
		\]
		and we would say that $x$ is beaten by $y$, we denote this as
		\[
			x \prec_v y
		\]
	\end{definition}

	\begin{definition}
		For a set of candidates $D \subseteq C$, for a preference list $v \in V$ and a preference profile $p \in P$ we denote $v$ and $p$ \emph{restricted to} $D$ by $v|_D$ and $p|_D$ respectively. $v|_D$ means $v$ after all the candidates who are not in $D$ have been removed from the preference list. $p|_D$ means that every preference list in $p$ has been restricted to $D$. For any sequence $v$, and $i \in \{1, \ldots, |v|\}$ we will denote by $v_{-i}$, $v$ with $v_i$ removed.
	\end{definition}

	\begin{definition}
		We define a \emph{poset}, or \emph{partially ordered set}, to be $(X, \le)$ where $X$ is a set, and $\le$ is a binary relation on $X$. $\le$ is also called a partial ordering because of the fact that not every pair of elements in $X$ needs to be related by $\le$, as opposed to a total ordering which must relate every pair.
	\end{definition}

	\begin{definition}
		For any poset $(P, \le)$, an \emph{upper bound} of a subset $X \subseteq P$ is an element $a \in P$ such that $a \le x$ for every $x \in X$. A \emph{least upper bound} is an \emph{upper bound} that is less than or equal to every other \emph{upper bound}. We denote this \emph{least upper bound} as $\sup_P X$ calling it the \emph{supremum} \cite{birkhoff1967lattice} and also as $\bigjoin_P X$ calling it the \emph{join}. When $X$ contains only two elements, we can use the join as a binary operator: $\bigjoin_P \{a, b\} = a \join_P b$. When $P$ is obvious from context we will simply write $\sup X$ or $\bigjoin X$. If the supremum exists, it is unique because posets are antisymmetric. The supremum is the same as the infimum in the inverse order.
	\end{definition}

	\begin{definition}
		For any poset $(P, \le)$, a \emph{lower bound} of a subset $X \subseteq P$ is an element $a \in P$ such that $a \ge x$ for every $x \in X$. A \emph{greatest lower bound} is a \emph{lower bound} that is greater than or equal to every other \emph{lower bound}. We denote this \emph{greatest lower bound} as $\inf_P X$ calling it the \emph{infimum} \cite{birkhoff1967lattice} and also as $\bigmeet_P X$ calling it the \emph{meet}. When $X$ contains only two elements, we can use the meet as a binary operator: $\bigmeet_P \{a, b\} = a \meet_P b$. When $P$ is obvious from context we will simply write $\inf X$ or $\bigmeet X$. If the infimum exists, it is unique because posets are antisymmetric. The infimum is the same as the supremum in the inverse order.
	\end{definition}

	\begin{definition}
		A poset, $(P, \le)$, is a \emph{lattice} if for any $x, y \in P$ the meet and join of $x$ and $y$ both exist. Note that the meet and join are unique by definition (if they exist).
	\end{definition}

	\begin{definition}
		A binary relation $R$ on a set $X$ is \emph{transitive relation} if $\forall a,b,c \in X$
		\[
			(aRb \textrm{ and } bRc) \implies aRc
		\]
	\end{definition}

	\begin{definition}
		The \emph{transitive closure} of a binary relation $R$ on a set $X$ is the transitive relation $R^t$ on $X$ such that $R \subseteq R^t$ and $R^t$ is minimal \cite[p. 337]{lidl1998applied}.
	\end{definition}

	\begin{definition}
		For any poset $(P, \le)$, let $\sigma$ be a permutation of $P$. We define the \emph{inversions} of $\sigma$ to be a binary relation $\Inv_{\sigma}$ on $P$:
		\[
			\Inv_{\sigma} = \{(i,j) \mid i, j \in P, i < j, \sigma^{-1}(i) > \sigma^{-1}(j)\}
		\]
		We can read $i \Inv_{\sigma} j$ as ``$i$ is inverted with $j$ in $\sigma$''. $\Inv$ is a transitive relation because for any $i,j,k \in P$ if $i \Inv_{\sigma} j$ and $j \Inv_{\sigma} k$ then $i < j < k$ and $\sigma^{-1}(i) > \sigma^{-1}(k) > \sigma^{-1}(k)$ which means that $i \Inv_{\sigma} k$.

		In addition, let $(X, \le')$ be a lattice such that the elements of $X$ are permutations of $P$. For any $\sigma, \pi \in X$ we have
		\[
			\Inv_{\sigma \meet \pi} = (\Inv_{\sigma} \cup \Inv_{\pi})^t
		\]
		\cite{markowsky1994permutation}.
	\end{definition}

	\begin{definition}
		For any poset $(P, \le)$, let $x,y \in P$. We say that $x$ is a \emph{predecessor} of $y$ if $x < y$. We say that $x$ is a \emph{direct predecessor} of $y$ if $x$ is the greatest predecessor of $y$.
	\end{definition}

	\begin{definition}
		For any poset $(P, \le)$, let $x,y \in P$. We say that $x$ is a \emph{successor} of $y$ if $x > y$. We say that $x$ is a \emph{direct successor} of $y$ if $x$ is the least successor of $y$.
	\end{definition}

	\begin{definition}[$X^{ij}, \le^{ij}$]
		Let $(X, \le)$ be a lattice whose elements are permutations of a set $Y$. For any $i,j \in Y$ we define
		\[
			X^{ij} = \{ x \in X \mid x^{-1}(i) < x^{-1}(j) \}
		\]
		We then define the partial ordering, $\le^{ij}$, over $X^{ij}$ such that for $x, y \in X^{ij}$:
		\[
			x \le^{ij} y \iff x \le y
		\]
	\end{definition}

	\begin{definition}[$\le_s$]
		Let $(P, \le)$ be a poset and let $X$ be the set of all permutations on $P$. We define the partial ordering $\le_s$ on $X$ such that for all $\sigma, \pi \in X$:
		\[
			\sigma \le_s \pi \iff \Inv_{\sigma} \subseteq \Inv_{\pi}
		\]
	\end{definition}

	\begin{lemma}
		\label{identified-permutation-lattice-join}
		Let $(X, \le_s)$ be a lattice whose elements are permutations of a set $Y$, and $\le_s$ is defined as above. Let $\Inv$ be the inversion binary relation over $Y$ as defined above. Let $\join$ and $\join^{ij}$ denote the join in $(X, \le_s)$ and $(X^{ij}, \le^{ij}_s)$ respectively. For any $i,j \in Y$, if $i$ is either a direct successor or a direct predecessor of $j$ according to $\le_s$, then for all $x, y \in X^{ij}$:
		\[
			\exists(x \join y) \implies \exists(x \join^{ij} y)
		\]
	\end{lemma}

	\begin{proof}
		Assume $\exists(x \join y)$. Let $z = x \join y$. Then $z$ is an upper bound of $\{x, y\}$:
		\[
			z \ge_s x \textrm{ and } z \ge_s y
		\]
		And $z$ is the least upper bound of $\{x, y\}$: for every $a \in X$:
		\[
			(a \ge_s x \textrm{ and } a \ge_s y) \implies z \le_s a
		\]
		Since $x \in X^{ij}$, then $(i, j) \notin Inv_x$. Since $z \ge_s x$, then $(i, j) \notin Inv_z$, so $z \in X^{ij}$. By definition $z \ge_s x \implies z \ge^{ij}_s x$ and $z \ge_s y \implies z \ge^{ij}_s y$. Therefore $z$ is an upper bound of $\{x, y\}$ in $X^{ij}$.

		For any $a \in X^{ij}$ if $a$ is an upper bound of $\{x, y\}$ in $X^{ij}$ then clearly $a$ is also an upper bound of $\{x, y\}$ in $X$. Therefore $z \le_s a$, so $z \le^{ij}_s a$, which means $z = x \join^{ij} y$. So clearly $x \join^{ij} y$ exists.
	\end{proof}

	\begin{lemma}
		\label{identified-permutation-lattice-meet}
		Let $(X, \le_s)$ be a lattice whose elements are permutations of a set $Y$, and $\le_s$ is defined as above. Let $\Inv$ be the inversion binary relation over $Y$ as defined above. Let $\meet$ and $\meet^{ij}$ denote the join in $(X, \le_s)$ and $(X^{ij}, \le^{ij}_s)$ respectively. For any $i,j \in Y$, if $i$ is either a direct successor or a direct predecessor of $j$ according to $\le_s$, then for all $x, y \in X^{ij}$:
		\[
			\exists(x \meet y) \implies \exists(x \meet^{ij} y)
		\]
	\end{lemma}

	\begin{proof}
		Assume $\exists(x \meet y)$. Let $z = x \meet y$. Then $z$ is a lower bound of $\{x, y\}$:
		\[
			z \le_s x \textrm{ and } z \le_s y
		\]
		And $z$ is the greatest lower bound of $\{x, y\}$: for every $a \in X$:
		\[
			(a \le_s x \textrm{ and } a \le_s y) \implies z \ge_s a
		\]

		We will now detour to show that $z \in X^{ij}$. Since $z = x \meet y$, then $\Inv_z = (\Inv_x \cup \Inv_y)^t$ \cite{markowsky1994permutation}. Because $x,y \in X^{ij}$ we know that $(i, j) \notin (\Inv_x \cup \Inv_y)$. Therefore, in order to have $(i, j) \in (\Inv_x \cup \Inv_y)^t$ we would need to have $(i, k) \in \Inv_x$ and $(k, j) \in \Inv_y$ for any $k \in Y$, which is impossible because $i$ is either a direct successor or a direct predecessor of $j$. Therefore $(i, j) \notin Inv_z$, so $z \in X^{ij}$.

		By definition $z \le_s x \implies z \le^{ij}_s x$ and $z \le_s y \implies z \le^{ij}_s y$. Therefore $z$ is a lower bound of $\{x, y\}$ in $X^{ij}$.

		For any $a \in X^{ij}$ if $a$ is a lower bound of $\{x, y\}$ in $X^{ij}$ then clearly $a$ is also a lower bound of $\{x, y\}$ in $X$. Therefore $z \ge_s a$, so $z \ge^{ij}_s a$, which means $z = x \meet^{ij} y$. So clearly $x \meet^{ij} y$ exists.
	\end{proof}

	\begin{proposition}
		\label{proposition-identification-is-lattice}
		Let $(X, \le_s)$ be a lattice whose elements are permutations of a set $Y$, and $\le_s$ is defined as above. Let $\Inv$ be the inversion binary relation over $Y$ as defined above. For any $i,j \in Y$, if $i$ is either a direct successor or a direct predecessor of $j$ according to $\le_s$, then $(X^{ij}, \le^{ij}_s)$ is a lattice.
	\end{proposition}

	\begin{proof}
		We know that $\exists(x \join y)$ and $\exists(x \meet y)$ because $(X, \le_s)$ is a lattice. Therefore by lemma \ref{identified-permutation-lattice-join} and lemma \ref{identified-permutation-lattice-meet} we have $\exists(x \join^{ij} y)$ and $\exists(x \meet^{ij} y)$ respectively. So $(X^{ij}, \le^{ij}_s)$ is a lattice, by definition of a lattice.
	\end{proof}

	\begin{proposition}
		\label{proposition-grid-is-lattice}
		Let $(X, \le)$ be a lattice. Let $X^n$ be the set of all $n$-tuples of elements of $X$. Let $\le^n$ be defined as: for all $x, y \in X$ and all $i \in \{1, \ldots, n\}$
		\[
			x \le^n y \iff x_i \le y_i
		\]
		$(X^n, \le^n)$ is a lattice.
	\end{proposition}

	\begin{proof}
		By definition of a lattice, $(X^n, \le^n)$ is a lattice if for any two elements $s, t \in S^n$, $s \join t$ exists and $s \meet t$ exists.

		First we show that $s \join t$ exists. We define $u \in X^n$ such that $u_i = s_i \join t_i$, $\forall i \in \{1, \ldots, n\}$, and we show that $u = s \join t$. Because $u_i = s_i \join t_i$, we have
		\[
			u_i \ge s_i \textrm{ and } u_i \ge s_i
		\]
		so
		\[
			u \ge^n s \textrm{ and } u \ge^n t
		\]
		meaning that $u$ is an upper bound for $s$ and $t$. Suppose there is some $v \in X^n$ which is also an upper bound for $s$ and $t$. Then $\forall i \in \{1, \ldots, n\}$ we have
		\[
			v_i \ge s_i \textrm{ and } v_i \ge t_i
		\]
		so since $u_i = s_i \join t_i$, then $u_i \le v_i$. Therefore $u \le^n v$, i.e. $u$ is the least upper bound of $\{s, t\}$.

		Second we show that $s \meet t$ exists (by the same argument). We define $u \in X^n$ such that $u_i = s_i \meet t_i$, $\forall i \in \{1, \ldots, n\}$, and we show that $u = s \meet t$. Because $u_i = s_i \meet t_i$, we have
		\[
			u_i \le s_i \textrm{ and } u_i \le s_i
		\]
		so
		\[
			u \le^n s \textrm{ and } u \le^n t
		\]
		meaning that $u$ is a lower bound for $s$ and $t$. Suppose there is some $v \in X^n$ which is also a lower bound for $s$ and $t$. Then $\forall i \in \{1, \ldots, n\}$ we have
		\[
			v_i \le s_i \textrm{ and } v_i \le t_i
		\]
		so since $u_i = s_i \meet t_i$, then $u_i \ge v_i$. Therefore $u \ge^n v$, i.e. $u$ is the greatest lower bound of $\{s, t\}$.
	\end{proof}
